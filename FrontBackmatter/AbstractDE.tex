%*******************************************************
% Abstract
%*******************************************************
\pdfbookmark[1]{Zusammenfassung}{Zusammenfassung}
\chapter*{Zusammenfassung}
Das Entwerfen von Softwarearchitekturen, welche modernsten Anforderungen entsprechen sollen, ist eine komplexe und schwierige Aufgabe. Eine Möglichkeit Verteilte Software zu entwerfen bietet das Actor Model, welches im Zuge dieser Arbeit ausführlich diskutiert wird. Das in \cite{Hewitt1973AIntelligence} vorgestellte \textit{Actor-Model} wurde als Programmiermodel für parallelisierte Softwareentwicklung entworfen. Die Eigenschaften dieses Models wie die Nachrichtenorientierung oder die Datenisolierung, sind auch bestens als Grundlage für verteilte Softwarearchitekturen geeignet. \\
Um Softwareanwendungen zu entwickeln, welche den herausfordernden aktuellen Anforderungen gerecht werden, sollte die Architektur der Software verschiedene grundlegende Eigenschaften aufweisen. Unter anderem ist hier das \textit{Reactive Manifesto}, eine Grundlage moderner Softwarearchitektur, zu nennen. 
\\
Eine häufige Anforderung von Business Anwendungen ist die Konsistenz und Dauerhaftigkeit der verarbeiteten Daten. Die Verteilung solcher verteilten und Datenhaltenden Systeme birgt einige Probleme wie das CAP-Theorem \citep{gilbertPerspectiveCAPTheorem2012}. 
\\
Im Zuge der Masterarbeit, wurde aufgrund der theoretischen Auseinandersetzung mit den drei Hauptthemen Actor Model, Transaktionssyteme und Verteilte Systeme ein Anforderungskatalog erstellt, der die typischen Anwendungsgebiete dieser drei Themen zusammenfasst. Der Anforderungskatalog beschreibt das fiktive Flugbuchungssystem \textit{TyrolSky}. Dabei liegt der Schwerpunkt der Implementierung auf der Verteilung der gesamten Anwendung, ohne dabei Dateninkonsistenzen bei den Anwendungsspezifischen Daten zuzulassen. 
\\
Die Implementierung erfolgt auf Basis des Actor Framworks \textit{Akka.net} \citep{Akka.netCommunityAkka.NETDocumentation}. Die Implementierte Lösung bietet vier verschiedene Komponenten welche alle mehrfach instanziiert werden. Die Verteilung der Daten innerhalb dieser Komponenten wird von der Software automatisch vorgenommen. 
\\
Die Arbeit konnte zeigen, dass das Actor-Model für die Entwicklung von verteilen Systemen geeignet ist. Das Entwerfen von verteilten Systemen mit Transaktionsdaten bleibt aber eine Herausforderung. Bei der bewältigung dieser Hürden, kann das \textit{Actor-Model} den Entwickler unterstützen.

\cleardoublepage
