%*******************************************************
% Abstract
%*******************************************************
\pdfbookmark[1]{Zusammenfassung}{Zusammenfassung}
\chapter*{Zusammenfassung}
Das Entwerfen von Softwarearchitekturen, die den modernsten Anforderungen entsprechen sollen, ist eine komplexe und schwierige Aufgabe. Eine Möglichkeit, Verteilte Software zu entwerfen, bietet das Actor Model, welches im Zuge dieser Arbeit ausführlich diskutiert wird. Das in \cite{Hewitt1973AIntelligence} vorgestellte \textit{Actor-Model} wurde als Programmiermodel für parallelisierte Softwareentwicklung entworfen. Die Eigenschften dieses Models, wie die Nachrichtenorientierung oder die Datenisolierung, sind bestens als Grundlage für verteilte Softwarearchitekturen geeignet. \\
Um Softwareanwendungen zu entwickeln, soll die Architektur der Software verschiedene, grundlegende Eigenschaften aufweisen, um den herausfordernen aktuellen Anforderungen gerecht zu werden. Das Reactive Manifesto kann hier als moderen Softwarearchitektur aufgezählt werden.
Eine häufige Anforderung von Business Anwendungen ist die Konsistenz und Dauerhaftigkeit der verarbeiteten Daten. Die Verteilung solcher Systeme birgt einige Probleme, wie das CAP-Theorem \citep{gilbertPerspectiveCAPTheorem2012}. \\
Im Zuge der Masterarbeit wurde, aufgrund der theoretischen Auseinandersetzung mit den drei Hauptthemen Actor Model, Transaktionssyteme und Verteilte Systeme, ein Anforderungskatalog erstellt, der die typischen Anwendungsgebiete dieser drei Themen zusammenfasst. Der Katalog beschreibt das fiktive Flugbuchungssystem \textit{TyrolSky}, dass für Benutzer die Möglichkeit bietet, Flugtickets zu reservieren bzw. zu buchen. Dabei liegt der Schwerpunkt der Implementierung auf der Verteilung der gesamten Anwendung, ohne dabei Dateninkonsistenzen bei den anwendungsspezifischen Daten zuzulassen. \\
Die Implementierung erfolgt auf Basis des Actor Framworks \textit{Akka.net} (\cite{Akka.netCommunityAkka.NETDocumentation}). Die implementierte Lösung bietet vier verschiedene Komponenten, welche alle mehrfach instanziiert werden. Die Verteilung der Daten innerhalb dieser Komponenten wird von der Software automatisch vorgenommen.

\cleardoublepage
