%*******************************************************
% Abstract
%*******************************************************
\pdfbookmark[1]{Zusammenfassung}{Zusammenfassung}
\chapter*{Zusammenfassung}
Das Entwerfen von Softwarearchitekturen, die den modernsten Anforderungen entsprechen sollen, ist eine komplexe und schwierige Aufgabe. Eine Möglichkeit, Verteilte Software zu entwerfen, bietet das Actor Model, welches im Zuge dieser Arbeit ausführlich diskutiert wird. Das in \cite{Hewitt1973AIntelligence} vorgestellte \textit{Actor-Model} wurde als Programmiermodel für parallelisierte Softwareentwicklung entworfen. Die Eigenschaften dieses Models, wie die Nachrichtenorientierung oder die Datenisolierung, sind bestens als Grundlage für verteilte Softwarearchitekturen geeignet. \\
Um Softwareanwendungen zu entwickeln, soll die Architektur der Software verschiedene elementare Eigenschaften aufweisen, um den aktuellen Anforderungen gerecht zu werden. Unter anderem bietet das \textit{Reactive Manifesto} (\cite{reactiveManifesto}) dabei eine Grundlage für eine moderne Softwarearchitektur. 
Eine häufige Anforderung von Business Anwendungen ist die Konsistenz und Dauerhaftigkeit der verarbeiteten Daten. Die Verteilung solcher Systeme birgt einige Probleme, wie das CAP-Theorem \citep{gilbertPerspectiveCAPTheorem2012}. \\
Im Zuge der Masterarbeit wurde, aufgrund der theoretischen Auseinandersetzung mit den drei Hauptthemen Actor Model, Transaktionssysteme und Verteilte Systeme, ein Anforderungskatalog erstellt, der die typischen Anwendungsgebiete dieser drei Themen zusammenfasst. Als Anwendungsbeispiel dient dafür das fiktives Fluguchungssystem \textit{TyrolSky}. Dabei liegt der Schwerpunkt der Implementierung auf der Verteilung der gesamten Anwendung, ohne dabei Dateninkonsistenzen bei den anwendungsspezifischen Daten zu zulassen. \\
Die Implementierung erfolgt auf Basis des Actor Frameworks \textit{Akka.net} (\cite{Akka.netCommunityAkka.NETDocumentation}). Die implementierte Lösung bietet vier verschiedene Komponenten, welche alle mehrfach instanziiert werden. Die Verteilung der Daten innerhalb dieser Komponenten wird von der Software automatisch vorgenommen.
Die Arbeit konnte zeigen, dass das Actor-Model für die Entwicklung von verteilen Systemen geeignet ist. Das Entwerfen von verteilten Systemen mit Transaktionsdaten bleibt eine Herausforderung, jedoch kann das Actor-Model als Unterstützung zur einfacheren Bewältigung dienen.

\cleardoublepage
