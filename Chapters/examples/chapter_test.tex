\section{Test}
\newpage

To date, the vast majority of WLAN systems have been based on IEEE 802.11 successors such as IEEE 802.11b, IEEE 802.11g, and IEEE 802.11a. These systems basically provide physical-layer based throughput enhancements over the original standard. In 2010, the upcoming IEEE 802.11n standard that increases the data rate up to 600 Mbps was approved. Among physical-layer enhancements such as the multiple-input multiple-output  concept using several antennas and doubling the channel spacing to 40 MHz, the improvements for reaching high data rates are achieved by new MAC layer features such as Block Acknowledgements, Transmission Opportunities, and Direct Link Setup. Besides the well-deployed standard enhancements, there exist other versions for Quality of Service support, transmit power control and dynamic frequency selection, security and encryption support, and for vehicular mesh networks. These standards, partly still under development, are meant to cover even more wireless applications. The great success of IEEE 802.11 as well as its widespread deployment, however, yields challenges caused, e.g., by a high utilization of the shared wireless medium due to the high number of participants. As a result, the overall goodput of the whole system may decrease due to interferences of the different wireless stations even if the physical data rate of every single participant increases. We conclude that technical possibilities and hardware production costs were crucial at the beginning of wireless networks. Today, however, interoperability, the number of participants, and protocol efficiency are the key factors that limit performance. These challenges will rise in the near future, which motivates the need of networks engineers for a simple, intuitive, and yet accurate framework to analyze, measure, and predict the performance regarding throughput, delay, jitter, and backlog in order to optimize existing and future wireless networks. The great success of IEEE 802.11 as well as its widespread deployment, however, yields challenges caused, e.g., by a high utilization of the shared wireless medium due to the high number of participants. As a result, the overall goodput of the whole system may decrease due to interferences of the different wireless stations even if the physical data rate of every single participant increases. We conclude that technical possibilities and hardware production costs were crucial at the beginning of wireless networks. Today, however, interoperability, the number of participants, and protocol efficiency are the key factors that limit performance. These challenges will rise in the near future, which motivates the 
