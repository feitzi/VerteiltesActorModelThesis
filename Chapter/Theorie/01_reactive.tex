\section{Reactive Manifesto}
\label{reactiveManifesto}
Das \textit{Reactive Manifesto} (siehe \cite{reactiveManifesto}) ist eine Sammlung an grundlegenden Anforderungen, welche eine verteilte Softwarearchitektur erfüllen soll, um die Verarbeitung großer Datenmengen in der Größenordnung von bis zu mehreren Petabytes in der Millisekunden bewerkstelligen zu können. 
\graffito{Ein Petabyte sind $10^{15}$ \textit{Bytes}, oder $1000$ \textit{Terabytes}}
Wie in \cite{reactiveManifesto} angegeben, ist das \textit{Reactive Manifesto} bestens geeignet um Cloud-Anwendungen zu erstellen, welche in mehreren Rechenzentren verteilt betrieben werden. \\
Die Autoren des Manifestes fordern von einer Anwendung vier grundlegende Eigenschaften damit sie dem \textit{Reactive Manifesto} entsprechen und somit als reaktive Anwendung angesehen werden kann. Das Manifest sieht folgende Anforderungen als Grundstein für eine erfolgreiche Softwarearchitektur, welche für verteilte Applikationen entworfen wird. 
\begin{description}
  \item[Antwortbereit]\label{reactivo:responsive}
  Ein reaktives System antwortet auf alle Anfragen immer innerhalb eines vorab definierten Zeitfensters. Der einzige Umstand der zu keiner Antwort führt, ist jener, wenn das System in einem Fehlerzustand ist. In allen anderen Fällen muss auf jede Anfrage innerhalb des Zeitfensters eine Antwort gesendet werden.
  Laut \cite{reactiveManifesto} ist diese Regel der Grundstein einer reactiven Applikation.
  \item[Widerstandsfähig]\label{reactivo:resilient}
  Bei einem Ausfall von Komponenten des Systems, wie Hard- oder Software, wird die Anforderung \textit{Antwortbereitschaft} des Systems nicht beeinflusst. Daher darf ein Ausfall einer Komponente keine Auswirkungen auf das Antwortverhalten der Applikation haben.\\
  Widerstandfähigkeit kann laut \cite{reactiveManifesto} durch Delegieren von Verantwortung an verschiedene Komponenten sowie das Replizieren von Komponenten erreicht werden. Weiters ist durch Isolation der Komponenten zu verhindern, dass ein Fehler einer Komponente auf andere Komponenten übergreift. Dadurch bleibt der Ausfall einer Komponente für das gesamte System ein überschaubares Problem. Ebenfalls ist es notwendig, beim Auftritt eines Fehlers im System die Behebung so schnell wie möglich durchzuführen, im Idealfall erfolgt dies automatisch.
  \item[Elastisch]\label{reactivo:elastic}
  Das Antwortverhalten einer Reactiven Anwendung darf durch einen plötzlichen Anstieg der eintreffenden Anfragen und der damit erhöhten Last nicht beeinträchtigt werden. Dies kann erreicht werden, indem, ähnlich wie bei der Wiederstandfähigkeit, auf Replizierung der Komponenten gesetzt wird. Jedoch wird für die Elastizität des System auf eine automatische Replikation der betroffenen Komponenten gesetzt. Vor dem Erreichen der Maximallast einer Komponente wird diese an einem anderen Ort repliziert und teilt sich somit die Last. 
  \item[Nachrichtenorientiert]\label{reactivo:messageDriven}
  Ein reactives System besteht, wie aus den vorhergegangenen Anforderung abzuleiten ist, aus mehreren gelösten Komponenten. Die Kommunikation zwischen den Komponenten muss, um als Reactive zu gelten, laut \cite{reactiveManifesto} über asynchrone Nachrichten erfolgen. Durch den Einsatz von Asynchronität ist eine Komponente während der Kommunikation mit einer anderen Komponente nicht von dieser abhängig. Weiters können Nachrichten ortsungebunden versendet werden. Dies ermöglicht somit die Skalierung von Komponenten. 
\end{description} 
Die vier Anforderungen Antwortbereitschaft, Widerstandfähigkeit, Elastizität und Nachrichtenorientiertheit müssen in einer Applikation zu jedem Zeitpunkt gewährleistet werden und in der Konzeptionsphase einer reaktiven Anwendung immer mit einfließen. Durch die Einhaltung dieser vier Grundrichtlinien kann eine Anwendung von vornherein auf Verteilung und Skalierung entworfen werden.\\

Die Anforderungen des \textit{Reactive Manifesto} ähneln der Definition des Actor-Models (siehe Abschnitt \ref{actor:definition}). Laut \cite{Vernon2015ReactiveAkka} ist das Actor-Model die führende Architektur bei der Implementierung und Umsetzung von reaktiven Anwendungen, jedoch schließt das reaktive Manifest andere Softwaremuster nicht aus. \\
Bei der Umsetzung einer Softwarearchitektur mit dem Actor-Model sind die Anforderungen Elastizität sowie Nachrichtenorientiertheit des \textit{Reactive Manifesto} implizit gegeben, da das Actor-Model diese, wie in den Kapiteln \ref{actor:definition} und  \ref{actor:Mailbox} beschrieben, explizit vorgibt.\\
Eine Applikation, welche korrekt mit dem Actor-Model umgesetzt wird, kann als Reactive Applikation angesehen werden wenn die Anforderung \textit{Widerstandfähigkeit} miteingebracht wird. Diese ist im Actor-Model nicht definiert, jedoch bieten moderne Implementierungen des Actor-Models diese Anforderung im Form von \textit{Supervision}, siehe hierzu Abschnitt \ref{actor:supervision}.
