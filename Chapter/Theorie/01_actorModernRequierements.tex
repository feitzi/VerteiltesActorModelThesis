\section{Zusätzliche Eigenschaften eines Actors}
Implementierungen des Actor-Models erweitern die Eigenschaften aus der Definition in Abschnitt \ref{actor:definition} noch um einige weitere Vorteile \citep{Vernon2015ReactiveAkka}. Diese sind nicht direkt der Definition eines Actors, wie von \cite{Hewitt1973AIntelligence} und \cite{Agha1985ActorsSystems} zuordenbar, aber für das Verständnis moderner Anwendungen mit dem Actor-Model von Bedeutung.
\subsection{Hierarchischer Actor-Baum}
Actors können ihre Vorteile erst ausspielen wenn sie sich in einem Actor-System (siehe Abschnitt \ref{actor:actorSystem}) befinden. Jedoch ist dadurch nicht definiert wie sich das Actor-System organisiert. Im Prinzip besteht ein Actor-System aus einer unbegrenzte Anzahl an unterschiedlichen Actors die untereinander kommunizieren. Durch die Anwendung einer hierarchischen Struktur wird in diese Menge eine Gliederung gebracht. \\
Erstellt ein Actor, durch die Abarbeitung einer Nachricht einen neuen Actor, so ist dieser neue Actor ein Kind des Erstellers. Der Ersteller wird dann als Eltern-Actor des neuen Kind-Actors betrachtet. Durch dieses Muster kann ein Actor-System als ungerichteter Baum dargestellt werden, da jeder Actor maximal einen Eltern-Actor haben kann und eine unbegrenzte Anzahl an Kinder besitzen kann. \\
In der Abbildung \ref{fig:actor:actorHierarchySample}, ist eine Hierarchie eines kleines Actor-Systems zu sehen. Das abgebildete System besteht aus sieben Actors wobei drei Actors als Eltern fungieren und bis zu drei Kindern besitzen. 
\begin{figure}
  \centering
  \includegraphics[width=0.6\linewidth]{gfx/actor/actorHierarchy}
  \caption{Ein Beispiel eines hierarchischen Actor-Baumes Actor-Hierarchie-Baumes}
  \label{fig:actor:actorHierarchySample}
\end{figure}

\subsection{Supervision}\label{actor:supervision}
Damit die im \textit{Reactive Manifesto (siehe Kapitel \ref{reactiveManifesto})} geforderte Bedingung \textit{Widerstandsfähigkeit} erfüllt wird, bieten Actor-Frameworks, wie beispielsweise \textit{Akka} oder \textit{Orleans} (siehe nähere Framework Beschreibung unter Abschnitt \ref{sec:ActorFrameworks}) ein Fehlerhandling welches \textit{Supervision} genannt wird. Das, unter anderem von \cite{sargent2016play} beschriebene, Vorgehen ermöglicht es, mögliche Fehlerquellen zu kapseln und beim Auftreten eines Fehlers während der Laufzeit den Fehler unabhängig vom Rest des Systems zu lösen.\\
Das Beispiel in Abbildung \ref{fig:actor:actorHierarchySample}, enthält einen Prozessor-Actor \textit{Processor-A}, welcher zwei Kinder hat. Wenn es nun in einem der zwei Kind-Actors zu einem Fehler kommt, sollte der übergeordnete Actor \textit{Coordinator} im Idealfall nichts davon nichts mitbekommen. Daher kann in diesem Beispiel aus den zwei untersten Actors, wie in Abbildung \ref{fig:actor:actorHierarchySupervison} dargestellt, ein \textit{ErrorKernel} gebildet werden. Wenn innerhalb dieses \textit{Kernels} ein Fehler auftritt, wird der Eltern-Actor, was in diesem Beispiel der Actor \textit{Processor-A} ist, über den Fehler benachrichtigt und kann dann darauf entsprechend reagieren. Die Benachrichtigung über den Fehlerzustand wird in Abbildung \ref{fig:actor:actorHierarchySupervisonMessaging} dargestellt. Alle anderen Actors im System bekommen von diesem Vorgehen nichts mit und arbeiten ohne Unterbrechung weiter. \\
\begin{figure}
  \centering
  \begin{minipage}{.4\textwidth}
    \centering
    \includegraphics[width=\linewidth]{gfx/actor/actorHierchyErrorKernel}
    \caption{Einem Teil der Hierarchie aus Abbildung \ref{fig:actor:actorHierarchySample} wird ein \textit{Error-Kernel} zugewiesen.}
    \label{fig:actor:actorHierarchySupervison}    
  \end{minipage}%
  \begin{minipage}{.1\textwidth}
  \end{minipage}%
  \begin{minipage}{.5\textwidth}
   \centering
   \includegraphics[width=\linewidth]{gfx/actor/actorSupervisionMessageExample}
   \caption{Nachrichten-Austausch an den Supervisor des in Abbildung \ref{fig:actor:actorHierarchySupervison} definierten \textit{Error-Kernels}.}
   \label{fig:actor:actorHierarchySupervisonMessaging}
  \end{minipage}
\end{figure} 

\subsection{Location-Transparenz}\label{actor:locationTransparency}
Eine weitere Eigenschaft, welche etliche Implementierungen dem \textit{Actor-Model} hinzufügen, ist eine nicht gekoppelte Beziehung zwischen den Actors. Wie in Abschnitt \ref{actor:definition} erklärt, benötigt ein Actor für das Senden einer Nachricht an einen anderen Actor dessen Adresse. Mit dem Prinzip der \textit{Location-Transparenz} wird die Adresse eines Actors so gestaltet, dass sie nicht an einen bestimmten Prozess oder Rechner gebunden ist. Durch Entkoppelung ist es für den sendenden Actor nicht relevant wo sich der Empfänger befindet \citep{Vernon2015ReactiveAkka} und \citep{sargent2016play}.

