\chapter{Verteilte Systeme}
Mit dem Beginn des Computerzeitalters in den 50er Jahren galten Leistungsstarke Rechner als kostenintensiv, benötigten viel Platz und operierten meist unabhängig und autark von anderen Geräten. Durch die Weiterentwicklung von Hardware sowie auch Software konnte im lauf der Zeit diese Entwicklung von Geräten so weit gebracht werden, dass heute Computer kostengünstig sowie platzsparend sind. Weiters ist die Entwicklung der Netzwerktechnologien so weit fortgeschritten, dass eine vernetzung der Geräte keine all zu große Herausforderung mehr ist. Die Geräte jedoch optimal miteinander kommunizieren zu lassen stellt Entwickler von verteilten Systemen immer wieder vor neuen Herausforderungen. \citep{tanenbaum2007distributed} \\
Dieses Kapitel soll eine Zusammenfassung über Verteile Systeme geben und dessen Problematiken aufzeigen sowie  bekannte Lösungsvorschläge erarbeiten.

\section{Definition Verteilte Systeme}
Der Begriff \textit{Verteilte Systeme} fasst eine große Anzahl an Themenblöcken zusammen, demenstprechend gibt es auch eine mehrzahl an verschiedenen Definitionen welche sich zum Teil auch voneinander grob unterscheiden. Eine weitläufige und angesehene Definition ist jene von \cite{tanenbaum2007distributed}:
\begin{quote}
    Ein verteiltes System ist eine Ansammlung von unabhängigen Rechnern, die den Anwendern als ein einziges kohärentes System erscheint.
    \label{quote:distributedSystem:tanenbaum}
\end{quote}

