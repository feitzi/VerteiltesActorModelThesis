\chapter{Das Actor Model}
Das von \cite{hewitt1973session} erstmals vorgestellte theoretische Software Model, ermöglicht das Entwerfen und Erstellen von parallelisierten Softwaresystemen. Es unterstützt Softwareentwickler bei der effizienten Nutzung von Mehrkernprozessoren sowie bei Handhabung der  Komplexität mit welcher die Nebenläufige Programmierung einhergeht.   
\\
Neben geläufigen Modellen wie beispielsweise die Funktionale Programmierung oder die Objektorientierte Programmierung bietet das Actor Model eine Vorgehensweise Software zu schreiben. Dieses Kapitel wird die theoretischen Grundlagen des Actor Models aufzeigen und sich in drei Teile unterteilen. Zuerst wird das Actor Model in seiner Grundform wie von \cite{hewitt1973session} und \cite{Agha1985ActorsSystems} definiert gezeigt, anschließend werden Architekturkonzepte auf Basis des Actor Model erarbeitet und abschließend erfolgt eine Evaluierung des Actor Models.

\section{Actor Model Definition}
Ein Actor ist laut \cite{hewitt1973session} alles was in einem Software Skript benötigt wird um einen Programmablauf zu ermöglichen. Hierzu zählen Datenstrukturen, auch primitive Datentypen, logische Verknüpfungen und auch Benutzereingaben oder Berechnungsergebnisse.
All diese Bestandteile haben ein bestimmtes Verhalten und reagieren unterschiedlich auf äußere Einwirkungen.
\cite{hewitt1973session} zeigt das Formal alle Bestandsteile des Codes miteinander interagieren, in dem sie sich Nachrichten untereinander austauschen.  Weiters kapseln sie zusammengehörende Informationen und beinhalten das Verhalten dieses Wertes. Die Interaktion durch Nachrichten mit anderen Actors sowie das Kapseln des eigenen Verhalten zeichnet ein Actor aus.
% todo 
% nicht alles ist ein actor in praktischen umgebungen. das ist nur in der theorie so!!!!

\subsection{Actor}
Ein Actor ist eine logische Einheit welche Nachrichten empfangen kann. Für jede empfange Nachricht kann ein Actor auf eine oder mehrere der folgenden Arten reagieren \citep{Agha1985ActorsSystems} und \citep{Vernon2015ReactiveAkka}:
\begin{itemize}
    \item Eine oder mehrere Nachrichten an andere Actoren senden.
    \item Einen neuen Actor erstellen.
    \item Das eigene Verhalten (siehe \ref{actorBehaviour}) für zukünftige Nachrichten ändern.
    \item Den eigenen, privaten Zustand verändern
\end{itemize}
Es können auch mehrere dieser Aktionen pro empfangener Nachricht sequentiell durchgeführt werden. \\
Das erstellen von neuen Actors während der Behandlung einer empfangen Nachricht ermöglicht es, komplexe oder Zeitintensive Rechenaufgaben auf andere Actors auszulagern. Da die Abhandlung einer Nachricht, wie bereits erwähnt, pro Actor sequentiell erfolgt und innerhalb eines Actors keine direkte parallelität erlaubt ist



\subsection{Actor Verhalten}
\label{actorBehaviour}
Eine fundamentale Regel für einen Actor ist es das Verhalten sequentiell durchzuführen.
\subsection{}
