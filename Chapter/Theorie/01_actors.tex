\chapter{Das Actor Model}
Das von \cite{hewitt1973session} erstmals vorgestellte theoretische Software Model, ermöglicht das Entwerfen und Erstellen von parallelisierten Softwaresystemen. Es unterstützt Softwareentwickler bei der effizienten Nutzung von Mehrkernprozessoren sowie bei Handhabung der  Komplexität mit welcher die Nebenläufige Programmierung einhergeht.   //
Neben einhergehenden Modelen wie beispielsweise die Funktionale Programmierung oder die Objektorientierte Programmierung bietet das Actor Model viele Vorteile welche in diesem Kapitel erarbeitet werden.  Das Kapitel teilt sich in X Teile auf. Zuerst wird das Actor Model in seiner Grundform wie von \cite{hewitt1973session} und \cite{Agha1985ActorsSystems} definiert gezeigt, anschließend werden Grundlegende Konzepte mit dem Actor Model durchgegangen und abschließend erfolgt eine Evaluierung des Actor Models.

\section{Actor Model Definition}



Das Actor Model welches durch \cite{hewitt1973session} sowie \cite{Agha1985ActorsSystems} eingeführt wurde, ist eine Theoretisches Software Model. 
Es wurde von \cite{hewitt1973session} entworfen um die Entwicklung von paralleler Software zu vereinfachen und Mehrkern Prozessoren effizient zu verwenden.