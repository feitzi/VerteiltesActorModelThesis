\chapter{Das Actor Model}
Das von \cite{hewitt1973session} erstmals vorgestellte theoretische Software Model, ermöglicht das Entwerfen und Erstellen von parallelisierten Softwaresystemen. Es unterstützt Softwareentwickler bei der effizienten Nutzung von Mehrkernprozessoren sowie bei Handhabung der  Komplexität mit welcher die Nebenläufige Programmierung einhergeht.   //
Neben geläufigen Modellen wie beispielsweise die Funktionale Programmierung oder die Objektorientierte Programmierung bietet das Actor Model eine Vorgehensweise Software zu schreiben. Dieses Kapitel wird die theoretischen Grundlagen des Actor Models aufzeigen und sich in drei Teile unterteilen. Zuerst wird das Actor Model in seiner Grundform wie von \cite{hewitt1973session} und \cite{Agha1985ActorsSystems} definiert gezeigt, anschließend werden Architekturkonzepte auf Basis des Actor Model erarbeitet und abschließend erfolgt eine Evaluierung des Actor Models.

\section{Actor Model Definition}
Ein Actor ist laut \cite{hewitt1973session} alles was in einem Skript benötigt wird um einen Programmablauf zu ermöglichen. Hierzu zählen Datenstrukturen, auch primitive Datentypen, logische Verknüpfungen und auch Benutzereingaben oder Berechnungsergebnisse.  Der Thermus {Actor}
Ein Actor ist laut \cite{hewitt1973session}
