\chapter{Das Actor-Model} \label{actor:chapter}
Das von \cite{hewitt1973session} erstmals vorgestellte theoretische Software Model ermöglicht das Entwerfen und Erstellen von parallelisierten Softwaresystemen. Es unterstützt Softwareentwickler bei der effizienten Nutzung von Mehrkernprozessoren sowie bei der Handhabung der  Komplexität mit welcher die nebenläufige Programmierung einhergeht.   
% evt. ausbau um probleme mit nebeläufiger programmierung
\\
Neben geläufigen Modellen, wie beispielsweise die Funktionale Programmierung oder die Objektorientierte Programmierung, bietet das Actor-Model eine Vorgehensweise Software zu schreiben. Dieses Kapitel wird die theoretischen Grundlagen des Actor-Models aufzeigen und sich in drei Teile gliedern. Zuerst wird das Actor-Model in seiner Grundform, wie von \cite{hewitt1973session} und \cite{Agha1985ActorsSystems} definiert, gezeigt. Anschließend werden Architekturkonzepte auf Basis des Actor-Model erarbeitet und abschließend erfolgt eine Evaluierung des Actor-Models.
\section{Reactive Manifesto}
Das {Reactive Manifesto} (siehe \cite{reactiveManifesto}), ist eine Sammlung an Anforderungen welche Anwendungen erfüllen sollen um große Datenmengen in der größenordnung von bis zu mehreren Petabytes in Millisekunden zu bewerkstelligen. 
\graffito{Ein Petabyte sind  $10^{15}$ {Bytes}, oder $1000$ {Terabytes}}
Wie in \cite{reactiveManifesto} angegeben, ist das {reactive Manifest} bestens geignet um Cloud Anwendungen zu erstellen, welche in mehreren Rechenzentren verteilt betrieben werden. \\
Die Autoren des Manifestes fordern von einer Anwendung vier grundlegende Eigenschaften damit sie dem {Ractive Manifest} entsprechen und somit als {Reactive Anwendung} angesehen werden kann. Das Manifest sieht diese Anforderungen als Grundstein für eine erfolgreiche Software Architektur welche für verteilte Applikationen entworfen wird. 
\begin{description}
    \item{Antwortbereit}\label{reactivo:responsive}
    Ein Reaktives System Antwortet auf alle Anfragen immer innerhalb eines vorab definierten Zeitfensters. Der einzige Umstand der zu keiner Antwort führt, ist wenn das System in einem Fehlerzustand ist. In allen anderen Fällen muss auf jede Anfrage in des Zeitfensters eine Antwort gesendet werden.\\
    Laut \cite{reactiveManifesto} ist diese Regel der Grundstein einer {Reactiven} Applikation.
    \item{Widerstandsfähig}\label{reactivo:resilient}
    Bei einem Ausfall von Komponenten des Systems wie Hard- oder Software wird die Reaktive Anforderung {Antwortbereitsch} des Systems nicht beeinflusst. Daher darf ein ausfall einer Komponente keine Auswirkungen auf das Antwortverhalten der Applikation haben.\\
    Wiederstandfähigkeit kann laut \cite{reactiveManifesto} durch Delegieren von Verantwortung an verschiedene Komponenten sowie das Replizieren von Komponenten erreicht werden. Weiters ist durch Isolation der Komponenten zu verhindern das ein Fehler einer Komponente auf andere Komponenten übergreift. Dadurch bleibt der ausfall einer Komponente für das gesamte System ein überschaubares Problem. Ebenfalls ist es Notwendig, beim Auftretten eines Fehlers im Systems die Behebung dieses, im Idealfall automatisch, so schnell wie möglich durchzuführen.
    \item{Elastisch}\label{reactivo:elastic}
    Das Antwortverhalten einer Reactiven Anwendung darf auch durch einen plötzlichen Anstieg der eintreffenden Anfragen und somit der Last nicht beeinträchtigt werden. Dies kann erreicht werden das ähnlich wie bei der Wiederstandfähigkeit auf Replizierung der Komponenten gesetzt wird. Jedoch wird für die Elastizität des System auf eine automatische Replikation der betroffenden Komponenten gesetzt. Vor dem erreichen der Maximallast einer Komponente, wird diese an einem anderen Ort Repliziert und teilt sich somit die Last. 
    \item{Nachrichtenorientiert}\label{reactivo:messageDriven}
    Ein Reactives System besteht, wie aus den vorhergegangenen Anforderung abzuleiten ist, aus mehreren gelösten Komponenten. Die Kommunikation zwischen den Komponenten muss, um als Reactive zu gelten, laut \cite{reactiveManifesto}  über asynchrone Nachrichten erfolgen. Durch den einsatz von asynchronität ist eine Komponente während der Kommunikation mit einer anderen Komponente nicht von dieser Abhängig. Weiters können Nachrichten ortsungebunden versendet werden und ermöglicht somit die Skalierung von Komponenten. 
\end{description} 

\section{Actor-Model Definition}\label{actor:definition}
In der ersten Definition des Actor-Models von \citep{hewitt1973session} wird ein Actor als kleinster Bestandteil einer Software beschrieben. Jeder Bestandteil eines Software Skriptes, welches benötigt wird um einen Programmablauf zu ermöglichen, ist demnach als Actor zu betrachten. Hierzu zählen Datenstrukturen,  primitive Datentypen, logische Verknüpfungen und auch Benutzereingaben oder Berechnungsergebnisse. All diese Bestandteile haben ein bestimmtes Verhalten und reagieren unterschiedlich auf äußere Einwirkungen. Alle Komponenten des Codes interagieren untereinander, indem sie sich gegenseitig Nachrichten senden und auf diese entsprechend reagieren.\\
Das Actor-Modell kann als mathematisches Modell für parallele Datenverarbeitung, welches Akteure als universelle Grundlage der parallelen Verarbeitung verwenden, gesehen werden. Im Gegensatz zu anderen Modellen wurde das Actor-Modell von der Physik unserer Umwelt inspiriert. \footnote{Auch in der realen Welt werden Informationen nur über Nachrichten untereinander ausgetauscht. Jede Interaktion zur Umwelt erfolgt, in dem man einen Impuls an ein anderes Objekt gibt. Wenn wir beispielsweise den Namen einer Personen wissen wollen, können wir diesen nicht einfach ablesen, vielmehr müssen wir die Person nach deren Namen fragen. Ist die Person gewillt dem Fragenden diese Auskunft zu erteilen, antwortet sie mit einer anderen Antwortnachricht darauf. In welcher Form diese Nachrichten übermittelt werden (Sprache, Brief, Gesten etc. ), ist nicht relevant.} \citep{Vernon2015ReactiveAkka} .  In modernen Implementierungen des Actor-Models ist ein Actor nicht der kleinste Bestandteil eines Skriptes, hier wird vielmehr versucht, wie unter \ref{actor:parallelism}  beschrieben, die kleinst mögliche Aufgabe einer Software als Actor abzubilden. \\

Weiters bündeln Actoren zusammengehörende Informationen und beinhalten ein Verhalten dieser.  Grundsätzlich lässt sich sagen, dass die Interaktion durch Nachrichten mit anderen Actors sowie das Kapseln des eigenen Verhalten einen Actor auszeichnet.
% * <feitzinger.magdalena@gmail.com> 2018-02-06T19:32:50.078Z:
% 
% das wort "kapseln" isch eher verwirrend. vasuach a anders wort zum finda. 
% 
% ^.

\subsection{Actor}
Ein Actor ist eine logische, atomare Einheit welche Nachrichten empfangen und bearbeiten kann. Nachrichten sind die einzige Möglichkeit wie Actors untereinander, sowie mit externen Komponenten, kommunizieren können \citep{Agha1985ConcurrentParallelism}. Nachrichten werden an einen Actor gesendet und von diesem sequentiell abgearbeitet. Dabei muss ein Actor folgende fünf Bedingungen erfüllen \citep{Agha1985ConcurrentParallelism}.

\begin{description}
    \item[1. Kein Zugriff von außen]\label{actor:requirements:shareNothing}
    Die interne Repräsentation eines Actors ist nur für den Actor selbst bestimmt. Es darf keine Möglichkeit geben Werte eines Actors von anderen Actoren oder sonstigen Codestellen zu lesen oder zu manipulieren. Änderungen und Lesezugriffe sind ausschließlich über Nachrichten möglich (siehe \ref{actors:messages}). 
    \item[2. Asynchrone Kommunikation]\label{actor:requirements:AsynchronCommunication}
    Die Kommunikation mit anderen Actors erfolgt asynchron und frei von Wartemechanismen. Wird eine Nachricht an einen anderen Actor gesendet, wird nicht auf eine Antwort von diesem gewartet. Somit ist eine zügige Abarbeitung einer Nachricht möglich.
    \item[3. Sequentielle Abarbeitung von Nachrichten]
    Jeder Actor kann exakt eine Nachricht zu gleichen Zeit abarbeiten. Alle anderen Nachrichten müssen in einer Warteschlange auf die Abarbeitung warten (siehe \ref{actor:Mailbox}).
    \item[4. Frei von Sperrmechanismen]
    Durch die Bedingungen \textit{1.} und  \textit{3.} ist es ausgeschlossen, dass ein Codeteil innerhalb eines Actors von außen, oder gar von einem anderen Thread, ausgeführt wird. Daher sind Sperrmechanismen, wie unter \ref{actor:parallelism} beschrieben, nicht nötig und dürfen innerhalb eines Actors nicht verwendet werden.
    \item[5. Kleinst mögliches Aufgabengebiet]
    Ein einzelner Actor soll ein möglichst kleines Aufgabengebiet behandeln und andere Aufgaben an andere Actors abgeben. Dadurch wird einerseits ein hoher Nachrichtendurchsatz gewährleistet und andererseits die Komplexität eines einzelnen Actors vermindert, was wiederum zur Verständlichkeit des Quellcodes führt.
\end{description}

\subsubsection{Arbeitsaufgaben und Nachrichten}\label{actors:messages}
In \cite{Agha1985ActorsSystems} wird beschrieben, dass in einem Actor-System eine beliebige Anzahl an zu bearbeitenden Arbeitsaufträgen vorliegt. Diese Aufgaben werden in einem Actor-System zwischen den Actors zugewiesen. Während der Abarbeitung einer Aufgabe können neue Aufgaben erstellt und in das Actor-System eingebracht werden. Eine Aufgabe wird im Kontext eines Actor-Systems wie folgt repräsentiert:
% * <feitzinger.magdalena@gmail.com> 2018-02-06T19:38:42.087Z:
% 
% isch des wort "zwischen" do richtig? 
% moasch du ned eher, dass die aufgaben den einzelnen actos zugeteilt werdn. 
% 
% ^.
\begin{description}
    \item[Identifikation] Jede Aufgabe bekommt eine eindeutige Identifikation, welche sie von anderen Aufgaben unterscheidbar macht.
    \item[Ziel Adresse] Ziel Adresse an welchen Actor die Nachricht gerichtet ist.
% * <feitzinger.magdalena@gmail.com> 2018-02-06T19:40:49.849Z:
% 
% du hosch do zwoa mol neabsanand zieladresse stoh. es tät i ändra. stell da satz um, oder formuliers andersch. 
% 
% ^.
    \item[Aufgabenihalt] Beinhaltet alle benötigten Informationen um diese Aufgabe abzuarbeiten
\end{description} 
Um eine neue Aufgabe zu erstellen, muss somit die Adresse bekannt sein an welche diese gerichtet wird. Hier werden von \cite{Agha1985ActorsSystems} drei mögliche Wege aufzeigt wie ein Actor eine solche Adresse während der Abarbeitung einer anderen Aufgabe erhalten kann.
\begin{description}
    \item[Bereits bekannt] Die Adresse eines anderen Actors ist durch die Bearbeitung früherer Aufgaben bereits bekannt.
    \item[In Inhalt enthalten] Im Inhalt der gerade bearbeitenden Aufgabe ist eine andere Adresse enthalten.
    \item[Neuer Actor] Während der Abarbeitung der aktuellen Aufgabe wird ein neuer Actor erstellt. Durch das Erstellen des Actors ist auch dessen Adresse dem Ersteller bekannt.
% * <feitzinger.magdalena@gmail.com> 2018-02-06T19:45:06.093Z:
% 
% Formulier da letschte satz andersch. du hosch do so oft erstellen/ersteller dina. des verwirrt. 
% 
% ^.
\end{description}
In aktuellen Interpretationen des Actor-Models, wie beispielsweise \cite{Vernon2015ReactiveAkka} oder \cite{Brown2016ReactiveAkka.net.}, wird jedoch im Kontext von Actors nicht mehr von Aufgaben sondern nur mehr von Nachrichten gesprochen. Diese enthalten jedoch ebenfalls immer eine Adresse sowie einen Nachrichteninhalt. Die Identifikation der Nachricht wird in modernen Systemen jedoch nicht mehr verwendet.   \\
In der \textit{Domain-Driven}-Theorie (siehe \cite{Evans2004Domain-drivenSoftware}) werden Nachrichten, welche zwischen zwei Actors versendet werden, in drei Kategorien unterteilt. Einerseits werden Befehle als \textit{Commands} bezeichnet und andererseits Ereignisse welche im System auftreten als \textit{Events} benannt.  Zusätzlich gibt es Nachrichten welche rein zur Übertragung von Daten dienen ohne ein  Verarbeitung der enthaltenden Daten zu verlangen. Eine solche Nachricht wird als \textit{Document} bezeichnet.  Ein \textit{Document} kann zum Beispiel eine Antwortnachricht auf eine \textit{Command} sein welches bestimmte Informationen anfordert.\\
Die sprachliche Unterteilung von Nachrichten in \textit{Commands}, \textit{Events}  und \textit{Documents} führt laut \cite{Evans2004Domain-drivenSoftware} zu einer besseren Verständlichkeit des Quellcodes. Technisch gesehen unterscheiden Sie sich zuerst nicht. Eine semantische Unterscheidung dieser Nachrichtenarten kann jedoch in speziellen Anwendungsgebieten Sinn ergeben. Man denke hier an beispielsweise an \textit{Event-Sourcing} (siehe \cite{betts2013CQRSEventSourcing}).

\subsubsection{Actor Verhalten}
\label{actorBehaviour}
Empfängt ein Actor eine Nachricht, welche für ihn bestimmt ist, so kann dieser auf mehrere unterschiedliche Arten reagieren. In \citep{Agha1985ActorsSystems} und \citep{Vernon2015ReactiveAkka} werden folgende Möglichkeiten eines Actors beschrieben:
\begin{description}
    \item[Neue Nachricht erstellen] Erfordert die Bearbeitung der empfangenen Nachricht neue Berechnungen, kann eine oder auch mehrere neue Nachrichten erstellt werden. Diese neuen Nachrichten können dann an andere Actors gesendet werden. Auch Resultate, welche sich durch die Bearbeitung der Nachricht ergeben, können so dem aufrufenden Actor zurückgesendet werden.
    \item[Actor erstellen] Sind Teile der aktuellen Nachricht nicht vom bearbeitenden Actor bearbeitbar, kann von diesem ein neuer Actor erstellt werden, welcher für diese Aufgabe besser geeignet ist. Diesem neuen Actor wird dann eine Nachricht mit der Teilaufgabe zugesendet.
    \item[Verhalten ändern] Das Resultat einer Bearbeitung kann sein, dass die Verarbeitung der nächsten Nachrichten an diesen Actor in einer anderen Art und Weise erfolgen muss. Daher kann der Actor sich für die nächste Nachricht ein neues Verhalten bestimmen. Somit wird die nächste eintreffende Nachricht mit dem neuen Verhalten des Actors verarbeitet.
    \item[Zustand verändern] Während der Abarbeitung einer Nachricht kann der Actor auch seinen eigenen privaten Zustand verändern.
\end{description}
Das erstellen von neuen Actors während der Behandlung einer empfangen Nachricht ermöglicht es, komplexe oder zeitintensive Rechenaufgaben auf andere Actors auszulagern. Da die Abhandlung einer Nachricht, wie bereits erwähnt, pro Actor sequentiell erfolgt und innerhalb eines Actors keine direkte Parallelität erlaubt ist, wird, wie in \ref{actor:parallelism} näher beschrieben, durch die Auslagerung an neue Actors ein Höchstmaß an Parallelisierung gewährleistet \citep{Agha1985ConcurrentParallelism}.

\subsubsection{Nachrichten Eingang}\label{actor:Mailbox}
Jeder Actor kann jederzeit an jeden anderen Actor in einem System eine Nachricht zukommen lassen. Da ein einzelner Actor jedoch zur gleichen Zeit nur exakt eine Nachrichten bearbeitend kann, muss gewährleistet sein, dass wartende Nachrichten nicht verloren gehen. Nachrichten an ein Actor werden nach dem \textit{FIFO-Prinzip} abgearbeitet. 
\graffito{Bei \textit{First In, First Out}, kurz \textit{FIFO}, wird immer die älteste Nachricht in einer Warteschlange behandelt.} Um dies zu gewährleisten, benötigt ein Actor eine Nachrichtenwarteschlange.\citep{Agha1985ActorsSystems}.  Dadurch ist gewährleistet, dass die Nachrichten nach der Reihenfolge ihrer Ankunft beim Actor abgearbeitet werden. Eine abgeschlossene Nachricht kann anschließend verworfen werden da sie nicht mehr benötigt wird. 

\subsubsection{Parallelität}\label{actor:parallelism}
Obwohl ein Actor ankommende Nachrichten nur sequentiell bearbeitend kann, ist das Actor-Model ideal für parallele Aufgaben.\citep{hewitt1973session} Durch das Weiterreichen von Subaktionen an andere Actors ist jedoch höchste Parallelität gegeben. In einem laufenden Actor-System sind meist eine Vielzahl an einzelnen Actors parallel am Nachrichten abarbeiten. Dadurch kann ein Mehrkernprozessor optimal genützt werden. \citep{Agha1985ActorsSystems} \\
Durch die gleichzeitige Bearbeitung von maximal einer Nachricht pro Actor sowie die Unterbindung von Lesezugriffen auf den eigenen, internen State durch andere Actors (siehe \ref{actor:requirements:shareNothing}), benötigt es keine Sperrmechanismen innerhalb eines Actors. Dadurch wird erstens der Sourcecode übersichtlicher und zweitens entfällt das aufwändige Allokieren von Sperrressourcen.\citep{Vernon2015ReactiveAkka}
Wird nun bedacht, dass ein Actor grundsätzlich nur jeweils für eine kleine Aufgabe optimiert ist und andere Aufgaben an andere Actors abgibt, wird zusammen mit dem Wegfall von Sperrmechanismen die Abarbeitungszeit für eine einzelne Nachricht pro Actor auf ein absolutes Minimum gebracht. \citep{Vernon2015ReactiveAkka} \\
Zu beachten ist bei diesem Konzept, dass bei der Abarbeitung einer Nachricht nicht auf eine andere Nachricht gewartet werden soll. Dies führt zu neuen Architekturmustern welche auszugsweise unter \ref{theory:actorArchitecture} beschrieben werden. Das Vermeiden von warten auf Antworten führt zu einer stark asynchronen Kommunikation, welche wiederum die Geschwindigkeit der Abarbeitung einer Nachricht erhöht.

\subsection{Actor-System}\label{actor:actorSystem}
Ein einzelner Actor alleine kann aufgrund seiner begrenzten Möglichkeiten nur relativ wenig bewirken. Der große Vorteil erwirkt das Actor-Model erst durch die Macht von vielen einzelnen Actors welche miteinander interagieren. Eine solche Bildung von vielen Actors wird als Actor-System bezeichnet.\citep{Agha1985ActorsSystems}
\subsubsection{Kommunikation außerhalb des Systems}
Nachrichten werden prinzipiell nur unter Actors innerhalb eines Actor-System ausgetauscht. In \cite{Agha1985ActorsSystems} wird erläutert, dass für die Kommunikation mit Komponenten außerhalb eines Actor-System sogenannte Rezeptionisten Actors benötigt werden.\\
Ein solcher Rezeptionist ist prinzipiell ein gewöhnlicher Actor, welcher jedoch darauf spezialisiert ist, mit Komponenten  zu kommunizieren, welche keine Actors darstellen. Dafür benötigt ein solcher Rezeptionist eine definierte Schnittstelle. Über diese Schnittstelle können dann fremde Komponenten (Webservices, GUI Implementierungen  usw...) Nachrichten in ein vorhandenes Actor-System einbringen. Weiters ist es auch über diese speziellen Actors möglich auf eine Antwort einer eingebrachten Nachricht zu warten. Die eigentliche Abarbeitung der externen Nachricht wird vom Rezeptionist an dafür zuständige Actors innerhalb des Actor-Systems geliefert \citep{Agha1985ActorsSystems}.\\
Neben der Möglichkeit vom Einsatz eines Rezeptionisten um mit externen Komponenten zu kommunizieren wird in \cite{Agha1985ActorsSystems} auch die Möglichkeit beschrieben mit externen Actor-System zu interagieren. Dies ist nötig um beispielsweise zwei getrennt entwickelte Actor-Systeme zu verbinden. Hierfür wird eine Menge an externen Actors definiert. Diese wissen zwar von der Existenz eines externen Actor-Systems, kennen aber dessen Implementierung nicht. Über definierte externe Actoren können so trotzdem Nachrichten zwischen zwei fremden Actor-Systemen ausgetauscht werden. \\
Ohne den Einsatz von zumindest eines Rezeptionist oder eines externen Actors kann ein Actor-System nicht mit der Umwelt interagieren, da es der einzige Weg ist Nachrichten und somit Informationen von außen zu empfangen und ebenso Ergebnisse nach außen zu geben. 

\section{Zusätzliche Eigenschaften eines Actors}
Die Definition des Actor-Model unter \ref{actor:definition} wird durch gängige implementierungen davon noch durch einige Eigenschaften erweitert \citep{Vernon2015ReactiveAkka}. Diese sind zwar nicht direkt der Definition eines Actors, wie von \cite{Hewitt1973AIntelligence} und \cite{Agha1985ActorsSystems} zuordenbar, sind aber für das Verständnis moderner Anwendungen mit dem Actor-Model von Bedeutung.
\subsection{Hierarchischer Actor Baum}
Actors können ihre Vorteile erst ausspielen wenn sie sich in einem Actor-System (siehe \ref{actor:actorSystem}) befinden. Jedoch ist dadurch nicht definiert wie sich das Actor-System organisiert. Im Prinzip ist ein Actor-System eine unbegrenzte Anzahl an unterschiedlichen Actors die untereinander Kommunizieren. Durch die Anwendung einer Hierarchischen Struktur wird in diese Menge eine Struktur gebracht. \\
Erstellt ein Actor, durch die Abarbeitung einer Nachricht, einen neuen Actor, so ist dieser neue Actor ein \textit{Child} des Erstellers. Der Ersteller selber wird dann als \textit{Parent} des neuen Actors (\textit{Child}) betrachtet. Durch dieses Muster kann ein Actor-System als ungerichteter Baum dargestellt werden da jeder Actor maximal einen \textit{Parent} haben kann und eine unbegrenzte Anzahl an \textit{Childs} besitzen kann. \\
In der Abbildung \ref{fig:actor:actorHierarchySample}, ist eine Hierarchie eines kleines Actor-Systems zu sehen. Das Abgebildete System besteht aus 7 Actors wobei drei Actors als \textit{Parent} fungieren und bis zu drei \textit{Childs} besitzen. 
\begin{figure}
    \centering
    \includegraphics[width=0.6\linewidth]{gfx/actor/actorHierarchy}
    \caption{Ein Beispiel eines Actor-Hierarchie-Baumes}
    \label{fig:actor:actorHierarchySample}
\end{figure}

\subsection{Supervision}\label{actor:supervision}
Damit die im \textit{Reactive Manifest (siehe \ref{reactiveManifesto})} geforderte Bedingung \textit{Wiederstandsfähigkeit} zu erfüllen, bieten Actor Frameworks, wie beispielsweise \textit{Akka} oder \textit{Orleans} (siehe nähere Framework Beschreibung unter \ref{sec:ActorFrameworks}) ein Fehlerhandling welches \textit{Supervision} genannt wird. Das, unter anderem in \cite{sargent2016play} beschriebene, Vorgehen ermöglicht es, mögliche Fehlerquellen zu kapseln und beim Auftretten eines Fehlers während der Laufzeit denn Fehler unabhängig vom Rest des Systemes zu lösen.\\
Das Beispiel in Abbildung \ref{fig:actor:actorHierarchySample}, enthält beispielsweise einen Prozessor Actor \textit{Processor-A}, welcher selber zwei \textit{Childs} hat. Wenn es nun in einem der zwei \textit{Child}-Actors zu einem Fehler kommt, sollte der übergeordnete Actor \textit{Coordinator} im Idealfall nichts davon nichts mitbekommen. Daher kann in diesem Beispiel aus den zwei untersten Actors, wie in Abbildung \ref{fig:actor:actorHierarchySupervison} dargestellt, ein \textit{Error Kernel} gebildet werden. Wenn innerhalb dieses \textit{Kernels} ein Fehler auftritt, wird der \textit{Parent}, was in dem Beispiel der Actor \textit{Processor-A} ist, über den Fehler benachrichtigt, und kann dann darauf entsprechend reagieren. Alle anderen Actors im System bekommen von diesem Vorgehen nichts mit, und Arbeiten ohne Unterbrechung weiter. \\
\begin{figure}
    \centering
    \begin{minipage}{.4\textwidth}
        \centering
        \includegraphics[width=\linewidth]{gfx/actor/actorHierchyErrorKernel}
        \caption{Ein Teil der Hierarchie aus Abbildung \ref{fig:actor:actorHierarchySample} wird ein \textit{Error-Kernel} zugewiesen.}
        \label{fig:actor:actorHierarchySupervison}        
    \end{minipage}%
    \begin{minipage}{.1\textwidth}
    \end{minipage}%
    \begin{minipage}{.5\textwidth}
      \centering
      \includegraphics[width=\linewidth]{gfx/actor/actorSupervisionMessageExample}
      \caption{Nachrichten Austausch an den Supervisor des in Abbildung \ref{fig:actor:actorHierarchySupervison} definiterten \textit{Error-Kernel}.}
      \label{fig:actor:actorHierarchySupervisonMessaging}
    \end{minipage}
\end{figure}  

\subsection{Location Transparenz}\label{actor:locationTransparency}
Eine weitere Eigenschaft welche etliche Implementierungen dem \textit{Actor-Model} hinzufügen, ist eine nicht gekoppelte Beziehung zwischen den Actoren. Wie in Abschnitt \ref{actor:definition} erklärt, benötigt ein Actor für das Senden einer Nachricht an einen anderen Actor dessen Adresse. Mit dem Prinzip der \textit{location Transparenz}, wird die Adresse eines Actors so gestaltet, das sie nicht an einen Bestimmten Prozess oder Rechner gebunden ist. Durch entkoppelung ist es für den sendenden Actor nicht relevant wo sich der Empfänger befindet. \citep{Vernon2015ReactiveAkka} und \citep{sargent2016play}



\section{Actor Grafik Legende}\label{actor:diagram:description}
Um den Ablauf in einem Actor-System besser zu verstehen, bietet sich die visuelle Darstellung in Form von Diagrammen an. In den Werken \cite{kuhn2017reactive} und \cite{Vernon2015ReactiveAkka} wurde eine einheitliche, grafische Notation für das  Actor-Model vorgeschlagen, welche auch für diese Arbeit herangezogen wird. Im Folgenden nun eine kurze Einführung in diese Notation, welche die für diese Arbeit notwendigen Notationen enthält und kein Anspruch auf Vollständigkeit erhebt.\\
\subsection{Actors darstellen}
Ein einzelner Actor wird als Kreis dargestellt, innerhalb dieses Kreises steht der Name des Actors. In Abbildung~\ref{fig:actor:diagram:longLiveActor} wird ein Actor dargestellt welcher persistent existiert. Im Gegensatz dazu steht ein Actor welcher nur für eine kurze Zeit existiert; dieser wird mit gestrichelter Linien, siehe Abbildung \ref{fig:actor:diagram:shortLiveActor}, dargestellt. Ein Actor welcher nur für kurze Zeit existiert, wird meist verwendet um eine einzelne spezielle Nachricht abzuarbeiten. 
\begin{figure}
  \centering
  \begin{minipage}{.5\textwidth}
   \centering 
   \includegraphics[width=.5\linewidth]{gfx/actor/longLiveActor}
   \captionof{figure}{Ein einfacher Actor.}
   \label{fig:actor:diagram:longLiveActor}
  \end{minipage}%
  \begin{minipage}{.5\textwidth}
   \centering
   \includegraphics[width=.5\linewidth]{gfx/actor/shortLiveActor}
   \captionof{figure}{Ein Actor welcher direkt nach der Abarbeitung einer Nachricht wieder zerstört wird.}
   \label{fig:actor:diagram:shortLiveActor}
  \end{minipage}
\end{figure} 

\subsection{Nachrichten erstellen}
Das Erstellen einer Nachricht durch den Actor \textit{A} sowie die Zustellung durch den Actor \textit{A} an den Actor \textit{B} wird in Abbildung~\ref{fig:actor:diagram:simpleCreateAndSendMessage} abgebildet. Wichtig ist hier, dass der Actor \textit{A} nicht nur die Nachricht sendet, sondern sie auch erstellt.
\begin{figure}
  \centering
  \includegraphics[width=\linewidth]{gfx/actor/simpleCreateAndSendMessage}
  \caption{Actor \textit{A} erstellt eine neue Nachricht und sendet sie an Actor \textit{B}.}
  \label{fig:actor:diagram:simpleCreateAndSendMessage}
\end{figure}

\subsection{Actors erstellen}
Das Erstellen von Actors durch einen Actor wird durch zwei Kreise abgebildet. In Abbildung~\ref{fig:actor:diagram:childActorCreation} erstellt der Eltern-Actor einen neuen Actor, den sogenannten Kind-Actor.
\begin{figure}
  \centering
  \includegraphics[width=.7\linewidth]{gfx/actor/childActorCreation}
  \caption{Ein Actor erstellt einen neuen Actor.}
  \label{fig:actor:diagram:childActorCreation}
\end{figure}

\subsection{Darstellung eines Ablaufes}
Um einen konkreten Ablauf eines Actors abbilden zu können, benötigt man die Darstellung der  Reihenfolge des Eingangs der Nachrichten. Diese wird von \cite{kuhn2017reactive} und \cite{Vernon2015ReactiveAkka} als Linie dargestellt auf welcher sich Kreise befinden. Die Abarbeitung wird von oben nach unten dargestellt. In Abbildung~\ref{fig:actor:diagram:asynchronMessageReceivment} bekommt der Actor zwei Nachrichten zugesendet und sendet dazwischen eine Nachricht. Durch diese Notation ist es möglich sequentielle Abläufe darzustellen.
\begin{figure}
  \centering
  \includegraphics[height=6cm]{gfx/actor/actorAsynchMessgeFlow}
  \caption{Asynchroner Ablauf eines Actors.}
  \label{fig:actor:diagram:asynchronMessageReceivment}
\end{figure}

\section{Actor basierte Architekturmuster}
\label{theory:actorArchitecture}
\section{Evaluierung}
%todo
%Agha zitate kontrollieren