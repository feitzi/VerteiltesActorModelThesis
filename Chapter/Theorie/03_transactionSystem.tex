\chapter{Transaktionssysteme}
Die Verarbeitung von Informationen in einem Rechnernetzwerk stellt je nach Anwendungsfall verschiedenste Ansprüche an das verarbeitende System. Eine Transaktion selbst ist nach \cite{rahm1993HochleistungsTransaktionssysteme} eine Gruppe von zusammengehörenden Operationen welche gemeinsam das \textit{ACID} Prinzip, siehe Abschnitt \ref{sec:transactionTheory:acid}, einhalten und somit keine Datenanomalien oder Inkonsistenzen zulassen. \\
Durch den rapiden Anstieg an Anfragen an ein Transaktionssystem sowie die dadurch erfolgte Verteilung der Systeme, wie in \cite{li2017research} beschrieben, haben sich die Ansprüche an Transaktionssysteme in den letzten Jahren verändert. Die Verteilung von Datenverarbeitenden und Persistierenden Systemen führt zu neue Herausforderungen und andere Kriterien, nicht zuletzt wegen dem \textit{CAP-Theorem}, siehe Kapitel \ref{sec:distributedSystems:capTheorem}, welche in diesem Kapitel Diskutiert werden. 

\section{Transaktionskonzepte}
Ein Transaktionssystem muss nach \cite{rahm1993HochleistungsTransaktionssysteme}, einem definierten Konzept folgen und dieses jederzeit Garantieren können, um so Transaktionen zu verarbeiten. Dabei haben sich zwei Grundlegende Konzepte, \textit{ACID} und \textit{BASE}, entwickelt, auf welche Nachfolgend eingegangen wird. Welches der beiden Konzepte sich in der Praxis besser verwendet lässt, hängt vom Anwendungsfall ab. Weiters sind diese Konzepte nach \cite{EdlichFriedlandHampeBrauer201010} auch nicht ganz klar miteinander vergleichen, da sie teils unterschiedliche Ziele haben.

\subsection{ACID}\label{sec:transactionTheory:acid}
Der Begriff \textit{ACID} steht für die vier Eigenschaften \textit{Atomarität}, \textit{Konsistenz}, \textit{Isolation} und \textit{Dauerhaftigkeit}. Ein \textit{ACID} basierendes System gilt als restriktives Transaktionssystem welches durch die immer geltenden Bedingungen von allen vier Eigenschaften nur äußert schwer verteil-, und skalierbar ist \cite{PritchettBASE}.\\
Ein System welches \textit{ACID} erfühlt bietet laut \cite{haerder198Acid} für eine Transaktion immer folgende vier Eigenschaften.

\begin{description}
    \item[Atomarität] Alle Operationen welche sich innerhalb einer Transaktion befinden werden atomar ausgeführt. Das Bedeutet es wird entweder alles ausgeführt oder garnichts. Um das zu ermöglichen, muss das Transaktionssystem gewährleisten können, dass bei einem Rollback der Transaktion bereits ausgeführte Operationen auch wieder zurückgesetzt werden kann. 
    \item[Konsistenz] Bei Beendigung einer Transaktion ist sichergestellt, das sich die Datenbank in einem Konsistenten zustand befindet. Auch wird eine Transaktion nur erfolgreich abgeschlossen wenn alle Integrationsbedingungen des darunterliegenden Datenschemas erfühlt sind. 
    \item[Isolation] Eine gerade in Ausführung befindliche Transaktion wird von einer anderen Transaktion nicht beeinflusst. Die Isolation der Transaktion bewirkt, dass sich jede Transaktion so verhalten kann, als wäre sie die einzige im gesamten System.
    \item[Dauerhaftigkeit] Wird eine Transaktion erfolgreich abgeschlossen, muss vom Transaktionssystem garantiert werden können, das sich die Daten der Transaktion dauerhaft im System befinden. Durch diese Eigenschaft kann sich der Ersteller der Transaktion, im Falle das diese erfolgreich Abgeschlossen wird, darauf verlassen, dass die Daten im System enthalten sind.
\end{description}
Um diese vier Eigenschaften auch in einem verteilten System zu Garantieren ist laut \cite{PritchettBASE} ein zwei Phasen Commit erforderlich. In diesem werden in einer erster Commit Phase alle Beteiligten Komponenten gefragt ob sie die änderungen durchführen können. Wird dies von allen bestätigt, beginnt die zweite Phase. Hier werden der Commit dann von allen beteiligten tatsächlich ausgeführt. Durch die Gewährleistung von Konsistenz über alle beteiligten Rechner ist demnach die Verfügbarkeit nach \textit{CAP} nicht garantiert und je nach Grad der Skalierung wird diese auch verringert.

\subsection{BASE}\label{sec:transactionTheory:base}
Ist jedoch die Verfügbarkeit in einem speziellen Anwendungsfall wichtiger als die jederzeit gültige Konsistenz wie in \textit{ACID} so muss eine anderes Transaktionskonzept verwendet werden. Das in \cite{PritchettBASE} beschriebene Konzept \textit{BASE} tauscht Verfügbarkeit durch Konsistenz indem es Konsistenz hinter der Verfügbarkeit einordnet. \\
\textit{BASE} steht für \textit{basically available, soft state, eventually consistent} und garantiert keine Konsistenz, versucht diese jedoch sie so gut wie möglich einzuhalten. Das wird durch die so genannte eventuelle Konsistenz erreicht. Demnach befindet sich ein System nicht immer in einem Konsistenten zustand, jedoch versucht es diesen Zustand mit abgleich von anderen Komponenten zu erreichen. Dadurch kann der Durchsatz einer Anwendung erhöht werden. Die Konsistenz der Daten bezieht sich jedoch bei \textit{BASE} ausschließlich auf die korrektheit der Daten und nicht, wie bei \textit{ACID}, auch auf die Integrität dieser \citep{EdlichFriedlandHampeBrauer201010}. Weiters bewirkt \textit{BASE} auch durch die Implementierung eines \textit{Soft State}, das geschriebene Daten nicht direkt auf allen beteiligten Komponenten verfügbar sind. Diese werden nach und nach auf die beteiligten Rechner gespielt was wiederum zu erhöhung der erreichbarkeit führt, da auf ein zwei Phasen Protokoll verzichtet werden kann \citep{EdlichFriedlandHampeBrauer201010}.

\section{Event Sourcing}


