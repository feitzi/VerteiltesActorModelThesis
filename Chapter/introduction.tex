\chapter{Einleitung}
\section{Motivation}
Die Konzeption und Umsetzung von modernen Applikationen welche eine vielzahl an Anfragen gleichzeitig verarbeiten müssen, stellt Entwickler seit jeher vor große Herausforderungen. Einerseits muss, in häufigen Fällen äußerst komplexe, Geschäftsprozesse und Geschäftsumfelder beachtet werden, und andererseits muss die technische Komplexität bewerkstelligt werden. In \cite{Vernon2015ReactiveAkka} und \cite{Evans2004Domain-drivenSoftware} wird bereits darüber diskutiert, dass der Einsatz von Datenbanksystemen, unterschiedliche Anwendungsapplikationen mit verschiedensten Frameworks und Aufgabengebiete sowie die Verteilung, teils über verschiedene Kontinente, dieser Komponenten, Softwarearchitekten immer wieder vor ihre Grenzen stoßen lässt. \\
Weiters ist die parallisierung von Anwendungen immer wichtiger. Denn das lange Zeit gültige Gesetz von Moor, siehe dazu \cite{moore1965moore}, welches den stetige Anstieg an Prozessorleistung garantiert, ist aufgrund von physikalischen und wirtschaftlichen Limitierungen praktisch nicht mehr gültig \cite{mann2000end}. Deshalb steigt in modernen, finanziell vertretbaren, Prozessoren die Taktfrequenz nur mehr gering. Stattdessen werden Prozessoren mit immer mehr Kernen ausgestattet womit parallelisierte Anwendungen möglich werden und somit auch die Verarbeitungszeit für, dementsprechend Entwickelte, Operationen sinkt.





\section{Ziel der Arbeit}

\section{Methodik}

\section{Abgrenzung}

\section{Aufbau}

