\chapter{Umsetzung eines Verteilen Transaktionalen Systems mit dem Actor Model} 
\label{cha:practicalDevelopment}

Während der Umsetzung des Anforderungskataloges aus Abschnitt \ref{sec:Eruierung:technicalRequierements} wurde versucht das erarbeitete Wissen aus den theoretischen Kapitel praktisch anzuwenden. Die implementierte Architektur des Flugbuchungssystem legt seinen Schwerpunkt auf die Verteilung des gesamten Systems. Mit der Nachfolgend vorgestellten Implementierung ist es möglich Anzahl an Instanzen der verschiedenen Komponenten Während des Betriebes beliebig anzupassen. \\
Das folgende Kapitel erklärt den Aufbau der praktischen Anwendung. Nach der Einführung des verwendeten Frameworks um mit dem Actor Model arbeiten zu können, wird zuerst der Komponentenorientierte Aufbau der Implementierung vorgestellt. Anschließend wird auf die Implementierung der Abfrage und Befehlverarbeitung eingegangen, welche auf die starke Orientierung auf Verteilung angepasst ist. Nachfolgend wird die Implementierung der Garantierten Nachrichtenzustellung aus Abschnitt \ref{sec:actor:patterns:guaranteedDelivery} vorgestellt. Abschließend wird das Kapitel die Testapplikation vorstellen mit welcher die verteilte Anwendung geprüft wurde sowie auf die während der Entwicklung verwendete Environment Umgebung einehen.

\section{Einführung in das \textit{Akka.net} Framework}
Das Framework \textit{Akka} stellt das Actor-Model für \textit{Java} und \textit{Scala} Entwickler zur verfügung. Bei Entwickler dieser Sprachen konnte sich das Framework, und somit auch das Actor-Model einen guten Namen machen. Für \textit{.Net} Entwickler mit den Sprachen \textit{C\#} und \textit{F\#} wurde das Projekt mit Hilfe der OpenSource Community unter dem Name \textit{Akka.Net} für die \textit{.Net} Plattform portiert. Somit sind die funktionalität und der Aufbau der Frameworks zwischen \textit{Akka} und \textit{Akka.net} identisch, wenngleich auch die Implementierung, aufgrund der Sprach und Laufzeitunterschiede zwischen der \textit{Java Virtual Machine (JVM)} und der \textit{Common Language Runtime (CLR)}, siehe hierzu \cite{JvmVsClrsinger2003jvm}, unterschiedlich ist. \\
Die Implementierung der hier beschriebenen Anwendung erfolgt für die \textit{CLR}, und somit mit Hilfe des Frameworks \textit{Akka.Net}. 

\subsection{Ein Actor in \textit{Akka.Net}}
Wie bereits erwähnt wird eine Anwendung auf Basis von \textit{Akka.Net} mit der Objektorientierten Sprache \textit{C\#} geschrieben. Somit werden auch Actors als Klassen repräsentiert. In \textit{Akka.Net} gibt dafür Basisklassen welche die Funktionalität eines Actors, wie die Nachrichtenübermittelung oder das Verhalten auf empfangene Nachrichten, beinhalten. Die Actor Klasse selbst bietet jedoch für andere Klassen keine eigenen Methoden zur verfügung. Somit ist der Zugriff von Außen, wie in Abschnitt \ref{actor:requirements:shareNothing} beschrieben, nicht möglich. \\

\subsubsection{Erstellung eines Actors}
\label{subsec:implementation:actorCreation}

\subsubsection{Nachrichten}\label{subsec:implementation:akkaMessaging}
Da ein Actor selbst keine Methoden anbietet erfolgt die Interaktion mit der Umwelt des Actors über Nachrichten. Dies wird in \textit{Akka.Net} durch Objekte sichergestellt. Eine Nachricht selbst muss in \textit{Akka.Net} keine bestimmte Struktur aufweisen, kann somit von jedem beliebigen Datentyp sein. Um jedoch die zustellung von Nachrichten über verschiedene Hosts zu gewähreisten, siehe dazu Abschnitt \ref{subsec:implementation:ApplicationDistribution}, sollte eine Nachricht serialisierebar sein. Dies ist jedoch für Nachrichten welche innerhalb einer Laufzeitinstanz übertragen werden nicht relevant. \\
Um nun eine Nachricht an einen Actor zu senden, wird die Referenz des Actors , siehe dazu Abschnitt \ref{subsec:implementation:ApplicationDistribution}, benötigt an welche die Nachricht gerichtet ist. Auf diese Referenz, welche ja selbst eine Instanz vom Typ \textit{ActorRef} ist, können nun für die übermittlung der Nachricht die Methoden \textit{Tell} oder \textit{Ask} aufgerufen werden.
\begin{description}
    \item[Tell] Möchte man eine Nachricht an einen Actor senden ohne auf dessen Antwort zu warten, so wie es das theoretische \textit{Actor Model} auch vorsieht, sollte die Methode \textit{Tell} verwendet werden. Die Methode selbst blockiert nicht, und die Nachricht wird in einem Thread an den Actor zugestellt. 
    \item[Ask] Die Methode \textit{Ask} blockiert den aktuellen Thread, bis eine Antwort auf die gesendet Nachricht zugestellt wird. Somit gleicht \textit{Ask} einem gewöhnlichen Methodenaufruf. Diese Variante verletzt jedoch die zweite Bedingung eines Actors aus Abschnitt \ref{actor:requirements:AsynchronCommunication}, welche besagt das die Kommunikation zwischen Actoren frei von Wartemechanismen sein sollte. Deshalb ist diese Variante für die zustellung von Nachrichten nur im Ausnahmefall zu empfehlen.  
\end{description}
Nachrichten welche über die Actor Referenz mit den eben beschriebenen Methoden zugestellt worden sind, werden in die Mailbox des Empfängers eingereiht. Enthält der Empfänger ein Verhalten für diesen Typ der Nachricht, so wird, sobald die Nachricht vom Actor aus der Mailbox genommen wird, diese Abgearbeitet. Enthält der Actor für diese Nachricht jedoch kein Verhalten, so wird die Nachricht verworfen, der Sender der Nachricht bekommt dies jedoch nicht mit. 

\begin{lstlisting}[caption=Versenden einer Nachricht an einen anderen Actor, label=code:actor:TellMethod]
    targetActorRef.Tell(new SpecificMessage());
\end{lstlisting}

\begin{lstlisting}[caption=Hier wird für den Actor \textit{MyTargetActor} das Verhalten für eine Einkommende Nachricht vom Typ \textit{SpecificMessage} festgellegt., label=lst:test]
    public sealed class MyTargetActor : ReceiveActor {
        public MyTargetActor() {
            Become<SpecificMessage>(x => {
                Console.Log("Nachricht von Typ SpecificMessage empfangen");
                //implement logic for handling SpecificMessage
            })
        }
    }    
\end{lstlisting}

\subsubsection{Erstellung von Actors}
Die Erstellung einer Instanz eines Actors erfolgt nicht über das Schlüsselwort \textit{new}, sondern wird vom Framework selbst übernommen. Dazu werden dem Framework alle Informationen zur verfügung gestellt welche es benötigt um den Actor zu erstellen, das beinhaltet beispielsweise den Typ des Actors, dessen Abängigkeiten sowie gegebenenfalls der Ort an welcher der Actor erstellt werden soll. Anschließend bekommt der aufrufer der erzeugung eine Referenz, die sogenannte \textit{ActorRef}, welche dazu dient mit dem erstellten Actor, über die Methoden aus Abschnitt \ref{subsec:implementation:akkaMessaging} zu kommunizieren. Eine direkte Referenz auf den Actor selbst wird nicht zur verfügung gestellt. \\
Durch die Verwendung einer eigenen, Framework basiertem Referenz anstatt einer direkten Referenz auf den Actor, können Actors ortsungebunden instanziert werden ohne das der Anwender des Actors dies im Code beachten muss. Die Verwendung eines Actors unterscheidet sich somit nicht davon ob er sich in der gleichen Laufzeitumgebung befindet oder ob er sich auf einem örtlich getrennten System befindet und die Verbindung zu diesem Actor über ein Netzwerk hergestellt wird. Die Informationen wo die einzelnen Actors wirklich instanziert werden, wird über eine Konfigurations Datei gesteuert. Somit kann das Verhalten des Systems einfach geändert werden und die Verteilung der gesamten Applikation wird erleichtert.

\section{Service Architektur}

\subsection{\textit{Lighthouse}}

\subsection{\textit{REST}-Schnittstelle}

\subsection{Abfragen}

\subsection{Domain-Service}

\section{Implementierung von Event Sourcing}

\section{Verteilung der Anwendung}
\label{subsec:implementation:ApplicationDistribution}

\subsection{Fixe Dienste}

\subsection{Verteilung transaktionaler Daten}
Erklärung von Sharding

\section{Garantierte Nachrichtenübermittelung}

\section{Externe Schnittstellen}

\section{Testapplikation}

\section{Verwendete Umgebung}
Welche Version, wie genau etc...
