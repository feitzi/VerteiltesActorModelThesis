\chapter{Umsetzung eines Verteilen Transaktionalen Systems mit dem Actor Model} 
\label{cha:practicalDevelopment}

Während der Umsetzung des Anforderungskataloges aus Abschnitt \ref{sec:Eruierung:technicalRequierements} wurde versucht, dass erarbeitete Wissen aus den theoretischen Kapitel praktisch anzuwenden. Die implementierte Architektur des Flugbuchungssystem legt seinen Schwerpunkt auf die Verteilung der gesamten Anwendung. Mit der Nachfolgend vorgestellten Implementierung ist es möglich, die Anzahl an Instanzen der verschiedenen Komponenten während des laufenden Betriebes beliebig anzupassen. \\
Das folgende Kapitel erklärt den Aufbau der praktischen Anwendung. Nach der Einführung des verwendeten Frameworks, welches verwendet wird um mit dem Actor Model arbeiten zu können, wird zuerst der Komponentenorientierte Aufbau der Implementierung vorgestellt. Anschließend wird auf die Implementierung der Abfrage und Befehlsverarbeitung eingegangen, welche auf die starke Orientierung auf Verteilung angepasst ist. Nachfolgend wird die Implementierung der Garantierten Nachrichtenzustellung aus Abschnitt \ref{sec:actor:patterns:guaranteedDelivery} vorgestellt. Abschließend wird das Kapitel die Testapplikation vorstellen mit welcher die verteilte Anwendung geprüft wurde sowie auf die, während der Entwicklung verwendete Environment Umgebung, eingehen.

\section{Einführung in das \textit{Akka.net} Framework}
Das Framework \textit{Akka} stellt das Actor-Model für \textit{Java} und \textit{Scala} Entwickler zur Verfügung. Bei Entwickler dieser Sprachen konnte sich das Framework, und somit auch das Actor-Model einen guten Namen machen. Für \textit{.Net} Entwickler mit den Sprachen \textit{C\#} und \textit{F\#} wurde das Projekt mit Hilfe der OpenSource Community unter dem Name \textit{Akka.Net} für die \textit{.Net} Plattform portiert. Somit sind die Funktionalität und der Aufbau der Frameworks zwischen \textit{Akka} und \textit{Akka.net} identisch, wenngleich auch die Implementierung, aufgrund der Sprach und Laufzeitunterschiede zwischen der \textit{Java Virtual Machine (JVM)} und der \textit{Common Language Runtime (CLR)}, siehe hierzu \cite{JvmVsClrsinger2003jvm}, unterschiedlich ist. \\
Die Implementierung der hier beschriebenen Anwendung erfolgt für die \textit{CLR}, und somit mit Hilfe des Frameworks \textit{Akka.Net}. 

\subsection{Ein Actor in \textit{Akka.Net}}
Wie bereits erwähnt, wird die Anwendung auf Basis von \textit{Akka.Net} mit der Objektorientierten Sprache \textit{C\#} geschrieben. Somit werden auch Actors als Klassen repräsentiert. In \textit{Akka.Net} gibt es dafür Basisklassen welche die Funktionalität eines Actors, wie die Nachrichtenübermittlung oder das Verhalten für empfangene Nachrichten, beinhalten. Die Actor Klasse selbst stellt jedoch für andere fremde Klassen  keine Methoden zur Verfügung. Somit ist der Zugriff von Außen, wie in Abschnitt \ref{actor:requirements:shareNothing} beschrieben, nicht möglich. \\

\subsubsection{Nachrichten}\label{subsec:implementation:akkaMessaging}
Da ein Actor selbst keine Methoden anbietet erfolgt die Interaktion mit der Umwelt des Actors über Nachrichten. Dies wird in \textit{Akka.Net} durch Objekte sichergestellt. Eine Nachricht selbst muss in \textit{Akka.Net} keine bestimmte Struktur aufweisen, kann somit von jedem beliebigen Datentyp sein. Um jedoch die zustellung von Nachrichten über verschiedene Hosts zu gewähreisten, siehe dazu Abschnitt \ref{subsec:implementation:ApplicationDistribution}, sollte eine Nachricht serialisierebar sein. Dies ist jedoch für Nachrichten welche innerhalb einer Laufzeitinstanz übertragen werden nicht relevant. \\
Um nun eine Nachricht an einen Actor zu senden, wird die Referenz des Actors , siehe dazu Abschnitt \ref{subsec:implementation:ApplicationDistribution}, benötigt an welche die Nachricht gerichtet ist. Auf diese Referenz, welche ja selbst eine Instanz vom Typ \textit{ActorRef} ist, können nun für die übermittlung der Nachricht die Methoden \textit{Tell} oder \textit{Ask} aufgerufen werden.
\begin{description}
    \item[Tell] Möchte man eine Nachricht an einen Actor senden ohne auf dessen Antwort zu warten, so wie es das theoretische \textit{Actor Model} auch vorsieht, sollte die Methode \textit{Tell} verwendet werden. Die Methode selbst blockiert nicht, und die Nachricht wird in einem Thread an den Actor zugestellt. 
    \item[Ask] Die Methode \textit{Ask} blockiert den aktuellen Thread, bis eine Antwort auf die gesendet Nachricht zugestellt wird. Somit gleicht \textit{Ask} einem gewöhnlichen Methodenaufruf. Diese Variante verletzt jedoch die zweite Bedingung eines Actors aus Abschnitt \ref{actor:requirements:AsynchronCommunication}, welche besagt das die Kommunikation zwischen Actoren frei von Wartemechanismen sein sollte. Deshalb ist diese Variante für die zustellung von Nachrichten nur im Ausnahmefall zu empfehlen.  
\end{description}
Nachrichten welche über die Actor Referenz mit den eben beschriebenen Methoden zugestellt worden sind, werden in die Mailbox des Empfängers eingereiht. Enthält der Empfänger ein Verhalten für diesen Typ der Nachricht, so wird, sobald die Nachricht vom Actor aus der Mailbox genommen wird, diese Abgearbeitet. Enthält der Actor für diese Nachricht jedoch kein Verhalten, so wird die Nachricht verworfen, der Sender der Nachricht bekommt dies jedoch nicht mit. 

\begin{lstlisting}[caption=Versenden einer Nachricht an einen anderen Actor, label=code:actor:TellMethod]
    targetActorRef.Tell(new SpecificMessage());
\end{lstlisting}

\begin{lstlisting}[caption=Hier wird für den Actor \textit{MyTargetActor} das Verhalten für eine Einkommende Nachricht vom Typ \textit{SpecificMessage} festgellegt., label=lst:test]
    public sealed class MyTargetActor : ReceiveActor {
        public MyTargetActor() {
            Become<SpecificMessage>(x => {
                Console.Log("Nachricht von Typ SpecificMessage empfangen");
                //implement logic for handling SpecificMessage
            })
        }
    }    
\end{lstlisting}

\subsubsection{Erstellung von Actors}
\label{subsec:implementation:actorCreation}
Die Erstellung einer Instanz eines Actors erfolgt nicht über das Schlüsselwort \textit{new}, sondern wird vom Framework selbst übernommen. Dazu werden dem Framework alle Informationen zur verfügung gestellt welche es benötigt um den Actor zu erstellen, das beinhaltet beispielsweise den Typ des Actors, dessen Abängigkeiten sowie gegebenenfalls der Ort an welcher der Actor erstellt werden soll. Anschließend bekommt der aufrufer der erzeugung eine Referenz, die sogenannte \textit{ActorRef}, welche dazu dient mit dem erstellten Actor, über die Methoden aus Abschnitt \ref{subsec:implementation:akkaMessaging} zu kommunizieren. Eine direkte Referenz auf den Actor selbst wird nicht zur verfügung gestellt. \\
Durch die Verwendung einer eigenen, Framework basiertem Referenz anstatt einer direkten Referenz auf den Actor, können Actors ortsungebunden instanziiert werden ohne das der Anwender des Actors dies im Code beachten muss. Die Verwendung eines Actors unterscheidet sich somit nicht davon ob er sich in der gleichen Laufzeitumgebung befindet oder ob er sich auf einem örtlich getrennten System befindet und die Verbindung zu diesem Actor über ein Netzwerk hergestellt wird. Die Informationen wo die einzelnen Actors wirklich instanziert werden, wird über eine Konfigurations Datei gesteuert. Somit kann das Verhalten des Systems einfach geändert werden und die Verteilung der gesamten Applikation wird erleichtert.

\section{Service Architektur}
\label{sec:implementation:serviceAndComponentOrientation}
Um dasFlugbuchungssystem zu realisieren wurde auf eine Komponenten und Serviceorientierte Architektur geachtet. Dazu wurden die Anforderungen selbst in einzelne Kategorien unterteilt und daraus verschiedene Komponenten abgeleitet. \\
Einerseits besteht die Applikation aus einer Schnittstelle zu den Endbenutzer welche über \textit{HTTP} Aufrufe Anfragen an das System durchführen können. Die Anfragen selbst können entweder in Abfragen oder in Kommandos unterteilt werden. Somit ist eine weitere Komponente auf Abfragen (\textit{Queries}) spezialisiert und eine weitere Komponente behandelt Kommandos welche Daten im System verändern können. Alle Prozessebezogenen Inhalte werden in der Komponente \textit{Domain Model} behandelt. In dieser befindet sich die Logik des Flugbuchungssystem welche auch für die einhaltung der Datenkonsistenz zuständig ist. \\
Abschließend wird noch eine Komponente benötigt welche diese Komponenten miteinander in Verbindung bringt. Ableitend aus der \textit{Akka.Net} Community wird diese \textit{Lighthouse} genannt. Somit existieren in der \textit{TyrolSky} Anwendung folgende Komponenten:
\begin{itemize}
    \item API
    \item Query-Service
    \item Command-Service
    \item Domain-Service
    \item Lighthouse
\end{itemize}
Von jeder dieser Komponenten kann es zur Laufzeit mehrere Instanzen davon geben. 
Jeder dieser Komponenten kann mehrfach Instanziiert werden und somit die Last verteilen. Für eine korrekt Arbeitendes System ist es erforderlich das von jeder Komponente mindestens eine Instanz am System teilnimmt. Jedoch können teile des System auch funktionieren wenn nicht alle Komponenten verfügbar sind. So ist es beispielsweise möglich das Abfragen auf das System durchführbar sind obwohl keine Instanz von \textit{Domain Service} oder \textit{Command-Service} verfügbar sind. Jedoch können dann Befehle wie Tickets kaufen nicht durchgeführt werden und würden zu einem Fehler beim Benutzer führen. \\
Nun wird auf die genauere Funktionsweise und zuständigkeiten der einzelnen Komponenten eingegangen.

\subsection{\textit{Lighthouse}}
\label{subsec:implementation:lighthouse}
Dieser Service ist der einfachste und beinhaltet keine Logik welche speziell für die \textit{TyrolSky} Anwendung gedacht ist. Die für die vorliegende Anwendung eingesetzte  Version ist eine Abwandlung des öffentlich zugänglichen Basiscodes einer Lighthouse, siehe hierzu \cite{lighthouse}. \\
Die Komponente fungiert, wie ihr Name schon suggeriert, als Orientierungshilfe für neue Instanzen welche am gesamten System teilnehmen möchten. Diese Melden sich bei einem oder mehreren Laufenden Instanzen von \textit{Lighthouse} an und bekommen somit die Informationen von anderen Komponenten welche sich bereits im System befinden mit. Dieser Prozess wird in Abschnitt \ref{subsec:implementation:gossip} genauer beschrieben.

\subsection{\textit{API}-Schnittstelle}
\label{subsec:implementation:apiComponente}
Anfragen von Endbenutzern werden über eine \textit{REST} Schnittstelle mittels \textit{HTTP} an das System übertragen. Diese Anfragen werden von der Komponente \textit{API} angenommen. Anschließend wird von dieser die Anfrage geprüft, und anschließend an eine verfügbare \textit{Query} oder \textit{Command} Komponente weitergeleitet. \\
Für jede ankommende Anfrage, wird ein eigener Actor gestartet welche ausschließlich für diese konkrete Anfrage zuständig ist. Der Actor wartet auf eine Antwort von der Angefragten \textit{CQRS}-Komponente und beendet nach Abschließend die bestehende \textit{HTTP}-Verbindung zum Benutzer mit dem Entsprechenden Ergebnis. Wird von der Anfragten Komponente innherhalb eines Zeitfensters keine Antwort geliefert wird die Anfrage zum Benutzer mit einem Fehler abgebrochen. \\
In der Grafik \ref{fig:implementation:apiActorModel} ist der Aufbau des Actor Models innerhalb der API Komponente ersichtlich. Darauf ist auch zu sehen das die API Komponente selbst über keine Komplexen Aufbau verfügt und die Actoren darin als Actor Rezeptionisten wie in Kapitel \ref{actor:actorSystem} fungieren und somit eine Schnittstelle zum eigentlichen Actor System selbst herstellen. \\
Das Verwenden von einem separaten Rezeptionisten pro Benutzeranfrage ermöglicht es, die weitere Behandlung im dahinterliegenden Actorsystem an unterschiedliche Komponenteninstanzen zu verteilen. Dadurch können gleiche Anfragen an die selbe API Instanz von unterschiedlichen Instanzen der zuständigen Komponente abgearbeitet werden. 
\begin{figure}
    \centering
    \includegraphics[width=0.5\linewidth]{gfx/implementation/apiActorModel}
    \caption{Actor Ansicht der API Komponente}
    \label{fig:implementation:apiActorModel}
\end{figure} 

\subsection{Query-Service}
Die Anwendung separiert wie bereits erwähnt Abfragen und Befehle nach dem \textit{CQRS} Prinzip auf welches in Kapitel \ref{sub:transaction:cqrs} eingegangen wird. Die Abfrage Seite wird in der Implementierung der \textit{TyrolSky} Anwendung in einer eigenen Komponente realisiert. \\
Um Anfrage bearbeiten zu können welche eine Abfrage an das System stellen, überwacht die Komponente relevante Events welche im Event Sourcing, siehe dazu \ref{subsec:implementation:eventSouring}, aufgetreten sind. \\
Für jeden Typ von Abfrage wird dazu ein vorbereitetes Ergebnis bereit gehalten. Wird ein relevantes Event im Event Sourcing gefunden, so wird das Ergebnis mit den neuen Eventdaten aktualisiert. Dadurch können bei eintreffenden Anfragen diese, durch das bereits vorbereitete Ergebnis, ohne weitere Abfragen an ein Datenbanksystem oder ähnliches, abgearbeitet werden. Jedoch sind die Resultate einer Abfrage nicht garantiert aktuell. Wird ein bereits eingetretendes Event zu spät in das vorbereitete Ergebnis miteingerechnet, bekommt der Anfragende das Ergebnis zurückgeliefert, welches vor dem Event gegolten hat. Deshalb wird diese Komponente nur für Anfragen von Benutzern verwendet und keine Applikatorischen Entscheidungen darauf aufgebaut.

\subsubsection{Aufbereitung von Resultaten}
\label{subsubsub:implementation:queryActorModel:resultPreparator}
Tretten neue Events auf, so wird die Resultatsliste aktualisiert. Wird eine neue Instanz der \textit{Query} Komponente gestartet, so muss diese zuerst alle im System befindlichen Events abarbeiten um ein aktuelles Ergebnis ausliefern zu können. Da dies bei einer vielzahl von Events Zeit und Ressourcen benötigt, werden aufbereitete Resultate selbst wieder abgespeichert. Dies passiert sporadisch nach einer bestimmten Menge an bearbeitenden Events. Wird das System neugestartet, kann auf das letzte abgespeicherte Resultat zurückgegriffen werden, und anschließend auf diesem weitergearbeitet werden. \\
Die Speicherung von Ergebnis führt jedoch zu Problemen wenn man die Verteilung der Komponente beachtet. Da jede Instanz seine eigenen Ergebnisse vorbereitet müssen diese auch eigenständig gespeichert werden. Ansonsten ergeben sich beim Schreiben als auch beim Lesen konflikte zwischen den einzelnen Instanzen. Um dies zu umgehen wird entweder pro Instanz eine eigene permanente Persistierung, wie beispielsweise eine Datenbank, eingerichtet. Jedoch erhöht dies den Wartungsaufwand der Komponente da für jede Instanz eine eigene Datenbank gehalten werden muss. \\

Eine weitere Möglichkeit ist die Persistierung der Ergebnis pro Komponente in einer globalen Datenbank, wobei jeder Erzeuger von Ergebnissen seine Resultate mit einer Identifikation abspeichert. Dafür ist es jedoch erforderlich das Erzeuger von Ergebnissen eine eindeutige Identifikation besitzen. Dafür wurde ein globaler Namensservice eingerichtet, der im gesamten verteilten System eindeutige Namen vergibt. Meldet sich eine Komponente vom System ab, werden alle Namen, welche an diese Komponente vergeben wurden, wieder freigegeben. Meldet sich eine Komponente wieder beim System an, so bekommt diese die wieder freigegebenen Identifikationen. Diese kann somit mit den Ergebnissen der vorherigen Komponente, welche sich nicht  mehr im System befindet, weiterarbeiten. Für die Umsetzung des globalen Namensservice wurde auf das Pattern \textit{Cluster Singelton}, welches in Abschnitt \ref{subsec:implementation:singeltons} erklärt wird. \\
Durch die Verwendung einer globalen Datenbank, sowie auf daszurückgreifen eines globalen Namensservice ist während dem Betrieb der Anwendung der Wartungsaufwand auf ein minimum beschränkt. Weiters kann durch diese Variante sichergestellt werden, das bei bei einem neustart des Systems die Instanzen mit bereits zuvor erarbeiteten Resultaten weiterarbeiten können. 

\subsubsection{Implementierte Abfrage}
In der umgesetzten Implementierung werden Abfragen zur verfügung gestellt welche dem Benutzer eine Übersicht über angebotente Flüge sowie den Status von Flugtickets geben. Insgesamt sind drei unterschiedliche Abfragen möglich:

\begin{enumerate}
    \item Status eines Flugtickets Abfragen
    \item Liste alle zur verfügung stehenden Flüge
    \item Passagierliste eines bestimmten Fluges
\end{enumerate}

Für die dritte Abfrage wurde jedoch eine andere Abfragevariante gewählt. Anstatt auf Events zu horchen, wird die Abfrage direkt an die entsprechenden Domaineninstanzen (siehe dazu Abschnitt \ref{subsec:implementation:domainService}) weitergegeben. Dies führt dazu, dass die Abfrage selbst länger dauert, da keine Ergebnisse vorgehalten werden. Jedoch ist die abfrage selbst zeitlich exakter, und der Aufwand in der Entwicklung geringer. \\
Das Beispiel zeigt, dass je nach Anwendungsfall entschieden werden sollte wie die Abfrage von Daten exakt implementiert wird.

\begin{figure}
    \centering
    \includegraphics[width=\linewidth]{gfx/implementation/QueringServiceActorModel}
    \caption{Der Aufbau des Actor Systems der Query Komponente von \textit{TyrolSky}.}
    \label{fig:implementation:queryActorModel}
\end{figure} 

\subsubsection{Strukturierung}
Für die abarbeitung einzelner Abfragen wird, ähnlich wie bei der Komponente \textit{API} aus Abschnitt \ref{subsec:implementation:apiComponente}, auch für die realisierung von Abfragen auf die Hilfe von kurzlebige Actoren gesetzt. Diese führen den eigentlichen Abfrageprozess durch, und verwenden dazu die enstprechenden Query Actoren. In Abbildung \ref{fig:implementation:queryActorModel} ist der Aufbau der Query Komponente in Form von Actoren zu sehen. Darin ist zu erkennen das für jeden Typ von Abfrage ein Actor existiert welcher die Abfragen bearbeitet, den soganannten \textit{Supplier} welche selbst einen Child Actor besitzt welche sich um die Aufbereitung der Abfragedaten wie in Abschnitt \ref{subsubsub:implementation:queryActorModel:resultPreparator} beschrieben, kümmert. \\
Jede einzelne Abfrage wird einem Worker Actor zugewissen, welcher nur für den Zeitraum der Abfrage selbst instanziert ist. Dieser Fragt die Daten vom zuständigen \textit{Supplier} ab. 

\subsection{Command-Service}
\label{subsec:implementation:commandService}
 Ähnlich wie beim \textit{Query-Service}, wird auch beim \textit{Command-Service} für jeden ankommenden Befehl ein eigener Actor gestartet, der für einen speziellen Typ von Anfrage zuständig ist. Es gibt auch hier für jeden möglichen Befehl einen eigenen Actor, welcher die Logik des Kommandos beinhaltet. Jedoch werden in der Implementierung der einzelnen Kommandos meist neue Befehle an einen oder mehrere zuständige Actors aus dem Domain-Service erzeugt und weitergeleitet. \\
% * <feitzinger.magdalena@gmail.com> 2018-05-18T06:03:36.815Z:
% 
% > Jedoch werden in der Implementierung der einzelnen Kommandos meist neue Befehle an einen oder mehrere zuständige Actors aus dem Domain-Service erzeugt und weitergeleitet
% Präzisieren! 
% 
% ^.
% * <feitzinger.magdalena@gmail.com> 2018-05-18T06:02:47.640Z:
% 
% > Ähnlich wie beim \textit{Query-Service}, wird auch beim \textit{Command-Service} für jeden ankommenden Befehl ein eigener Actor gestartet, der für einen speziellen Typ von Anfrage zuständig ist. Es gibt auch hier für jeden möglichen Befehl einen eigenen Actor welcher die Logik des Kommandos beinhaltet.
% Beide Sätze haben dieselbe Aussage. 
% 
% ^.
 In der vorliegenden Implementierung gibt es folgende unterschiedliche Typen von Kommandos, welche alle durch einen \textit{CommandHandler} repräsentiert werden:
 \begin{enumerate}
     \item Flüge erstellen
     \item Flug vorbereiten
     \item Ticket reservieren
     \item Ticket buchen
 \end{enumerate}
Die Abbildung \ref{fig:implementation:commandActorModel} zeigt den Aufbau des \textit{Command-Service}. Die  Geschäftslogik ist  in den Actoren des \textit{Domain Serice} beherbergt. Somit ist die Tätigkeit der einzelnen \textit{CommandHandler} darauf begrenzt, die betroffenen Actoren im \textit{Domain Serice} über die gewünschte Tätigkeit zu informieren und diese  auszuführen. 
 \begin{figure}
    \centering
    \includegraphics[width=0.8\linewidth]{gfx/implementation/CommandServiceActorModel}
    \caption{Der Aufbau des \textit{Command-Service} beinhaltet die Logik der einzelnen Befehle und leitete weitere Befehle an den \textit{Domain-Service} weiter }
    \label{fig:implementation:commandActorModel}
\end{figure} 



\subsection{Domain-Service}
\label{subsec:implementation:domainService} 
Die gesamte Geschäftslogik wie das verhalten von Flügen, Tickets und anderen Entitäten wird in der Komponente \textit{Domain-Service}  abgebildet. Die Actoren welche sich in dieser Komponente befinden, wurden so abgebildet das die Repräsentation der Anwendung selbst keine Rolle spielt. Das bedeutet, dass die Implementierung in dieser Komponente unabhängig davon umgesetzt wurde, ob die Anwendung selbst über das \textit{CQRS} Prinzip umgesetzt wird und ob die Interaktion mit dem Benutzer über eine \textit{HTTP} Schnittstelle umgesetzt wurde. \\
Die hier repräsentierten Entitäten, welche selbst ja wieder als Actoren repräsentiert werden, beinhalten neben der eigenen Logik auch einen Zustand welcher Persistiert wird. Dies wird durch \textit{Event Soucing} ermöglicht, welches in Abschnitt \ref{imp:subsec:implementation:eventSouring} genauer beschrieben wird. Ereignisse welche in der Entität selbst auftretten, werden persistiert und bei einem späteren Neustart des dazugehörigen Actors wieder eingespielt. Mit diesem Prinzip sind alle nachfolgend beschriebenen Entitäten ausgestattet. \\
Um eine Verteilung der Entitäten zu ermöglichen werden diese in Form von \textit{Sharding} auf unterschiedliche Hosts verteilt. \textit{Sharding} selbst wird in Kapitel \ref{subsec:implementation:ApplicationDistribution} genauer betrachtet. Für eine Verteilung mittels \textit{Sharding} müssen Actoren bestimmt werden, welche anschließend verteilt werden. Alle Kindelement dieser Actoren werden anschließend an den gleichen Ort wie ihre Parents verschoben. Innerhalb des \textit{Domain Service} gibt es folgende Actoren welche Verteilt werden:
\begin{itemize}
    \item Flug Nummer (\textit{FlightNumber})
    \item Flug Ticket (\textit{FlightTicket})
    \item Abbuchungs Koordination (\textit{ChargingCoordinator})
\end{itemize}
Auf die genaue Aufgabe sowie die Implementierung dieser drei Actoren wird nachfolgen eingegangen.

\section{Implementierung von Event Sourcing}
\label{sec:implementation:eventSouring}
Die Persistierung der Daten basiert auf dem \textit{Event Sourcing}, wie in Abschnitt \ref{sec:eventSourcing} beschrieben wird. Das Prinzip von \textit{Event Sourcing} ist es, aufgetretene Ereignisse abzuspeichern und gegebenenfalls wieder einzuspielen, um so einen Zustand reproduzieren und nachvollziehen zu können \citep{betts2013CQRSEventSourcing}. \\
Das verwendete Framework \textit{Akka.net} bietet eine grundlegende Unterstützung für \textit{Event Sourcing} \citep{Akka.netCommunityAkka.NETDocumentation}. Dabei werden Events als Nachrichten repräsentiert, welche dem Framework übergeben und anschließend abgespeichert werden. 
Erst nach erfolgreicher Speicherung des Events, kann auf das Event selber reagiert werden. Dadurch wird verhindert das Aktionen zu Events ausgeführt werden, jedoch das Event nicht persistiert werden kann. Weiters bietet \textit{Akka.Net} die Möglichkeit, Events zu einem späteren Zeitpunkt wieder einzuspielen und mit einer anderen Verhalten darauf zu reagieren als beim vorherigen Auftreten des Events. Dies ist erforderlich, damit Ereignisse welche in der Vergangenheit aufgetreten sind,  wieder eingespielt werden können. Dabei wird die Einspielung eines Events meist nicht gleicht behandelt, als wie wenn das Ereignisse gerade aufgetreten ist. Wird beispielsweise bei einer Ticketbuchung das Bankkonto belastet, und wird die Buchung nachträglich wieder eingespielt, sollte zwar der Status des Tickets aktualisiert werden, eine neuerliche Kontobelastung ist aber nicht mehr erforderlich. Deshalb werden in diesem Beispiel zwei Verhalten für das gleiche Ereignis benötigt.
\subsection{Aggregate Root}
Um die Implementierung der einzelnen Entitäten zu vereinfachen, wurde die Logik für \textit{Event Sourcing} in einer Basisklasse zusammengeführt, von welcher die eigentlichen Entitäts-Actoren ableiten. 
Die eigentliche Logik für die Speicherung des Events übernimmt die Methode \textit{Emit(IDomainEvent e, Action a)} die in derBasisklasse des Entitäts-Actors verfügbar ist. Der Verwender der Methode \textit{Emit()} übergibt dieser ein Event, welche alle Informationen über das aufgetretene Ereignis enthält. Weiters wird eine Action übergeben die ausgeführt wird, sobald das Event erfolgreich im \textit{Event Store} abgelegt wurde. Während der Speicherung des Events wird von \textit{Akka.net} sichergestellt, dass keine weiteren Nachrichten vom Actor abgearbeitet werden. Die Methode \textit{Emit()} speichert nicht nur das Event im \textit{Event Store} ab sondern ändert den Zustand des Actors und führt die übergebene Action aus. Dafür wird, nach der Speicherung des Events, die übergebene Action ausführt und die abstrakte Methode \textit{UpdateState()} aufgerufen. \\
Jede Ausprägung der abstrakten \textit{Aggregate Root} Klasse muss die Methode \textit{UpdateState(IDomainEvent e)} implementieren. Diese wird direkt nach der erfolgreichen Persistierung des Events von der Methode \textit{Emit()} aufgerufen. Die Implementierung der Methode \textit{UpdateState()} sollte den Zustand des Actors, entsprechend dem aufgetretenen Event, verändern. Im Gegensatz zu der zuvor übergebenen \textit{Action}, wird \textit{UpdateState()} auch bei einer historischen Einspielung der Events aufgerufen. Die Methode \textit{UpdateState()} soll mit allen unterschiedlichen Event Varianten, welche innerhalb des Actors auftreten, umgehen können und dementsprechend die interne Repräsentation des Actors verändern. Änderungen des internen Zustands des Actors, außerhalb der Methode \textit{UpdateState()}, führen bei einer Wiedereinspielung der Ereignisse zu unterschiedlichen Zuständen des Actors . Deshalb sollen sämtliche Änderungen des Zustands des Actors innerhalb der Methode \textit{UpdateState()} ausgeführt werden. \\
Nach dem Aktualisieren des Actors-State prüft die Methode ob ein \textit{Snapshot} ausgeführt werden soll, mehr dazu in Abschnitt \ref{subsec:implementation:eventSouring:Snapshot}. Abschließend wird die zuvor übergebene Methode \textit{Action} aufgerufen, welche die eigentliche Logik enthält, wie auf das aufgetretene Ereignis reagiert werden kann. Hier können andere Actors über das aufgetretene Ereignis benachrichtigt werden. Die Action wird nur aufgerufen, wenn das Event tatsächlich Auftritt, bei einer Wiedereinspielung des Events wird ausschließlich die Methode \textit{UpdateState} aufgerufen. \\ 

In der Abbildung \ref{fig:implementation:eventSourcingAggregateRoot} ist der beschriebene Prozess schematisch abgebildet. Darauf ist auch zu ersichtlich, dass zwei unterschiedliche Datenbanken verwendet werden, um Snapshots und Events abzubilden.
\begin{figure}
    \centering
    \includegraphics[width=\linewidth]{gfx/implementation/EventSourcingAkka}
    \caption{Ablauf der Speicherung eines Events innerhalb eines \textit{Aggregate Root} Actors mit \textit{Event Sourcing} und \textit{Snapshots}.}
    \label{fig:implementation:eventSourcingAggregateRoot}
\end{figure} 

\subsection{Snapshot}
\label{subsec:implementation:eventSouring:Snapshot}
Um bei einer Wiedereinspielung der Ereignisse für einen Actor nicht alle bereits aufgetretenen Ereignisse  wieder einspielen und dabei für jedes Ereignis die Methode \textit{UpdateState()} aufrufen zu müssen, wird regelmäßig ein Abbild des aktuellen Actorzustands gespeichert. Dies passiert innerhalb der Methode \textit{Emit()}. In der vorliegenden Implementierung wird nach jedem zwanzigsten Event ein Snapshot erstellt. Dafür wird die interne Datenrepräsentation eines Actors in eine eigene serialisierbare Klasse, den \textit{Actor State}, abgebildet. Diese Klasse wird auch bei der oben besprochenen Methode \textit{UpdateState()} verändert. Wird nun ein Snapshot erstellt, so wird der aktuelle \textit{Actor State} in einer dafür vorgesehenen Datenbank abgespeichert. Zusätzlich zum aktuellen \textit{State} wird die Sequenznummer des letzten Events, welches zu diesem \textit{State} geführt hat, gespeichert. Während der Erstellung des Snapshots kann der Actor selbst jedoch weiterhin Nachrichten abarbeiten. Dies beeinträchtigt nicht die Konsistenz der Datenrepräsentation. Durch die Zuordnung von einer Sequenznummer zu einem abgespeicherten \textit{State} werden Events, welche während der Erstellung des Snapshots anfallen, nicht mehr zum Snapshot dazu gezählt und somit später wieder auf Basis des vorhandenen Snapshots eingespielt. \\
Wird nun ein gestoppter Actor, dessen Zustand mittels \textit{Event Sourcing} persistiert wurde, wieder gestartet, so kommt es zur Wiederherstellung des vorherige Zustand. Während vergangene Ereignisse eingespielt werden, kann der Actor keine neuen Nachrichten verarbeiten. Diese verbleiben solange in der Mailbox des Actors bis die Einspielung abgeschlossen ist. \\
Die Wiederherstellung eines Actors wird erreicht, in dem zuerst der zuletzt verfügbare \textit{Snapshot} für den Actor in der Datenbank gesucht wird. Anschließend wird der Zustand des entsprechenden \textit{Snapshots} dem Actor als \textit{State} zugeteilt. Nun werden alle Events, welche nach der Erstellung des Snapshots angefallen sind, in der zeitlichen Reihenfolge ihres Auftretens wieder in den Actor eingespielt. Nach dem letzten eingespielten Event ist der Zustand des Actors exakt der gleiche wie vor dem Stoppen des Actors.
\subsubsection{Actors mit \textit{Event Sourcing}}
Wie bereits angedeutet, wird nicht von allen Actors innerhalb der \textit{TyrolSky}-Anwendung mittels \textit{Event Sourcing} der aktuelle Zustand gespeichert. Nur Actors, welche einen für die Konsistenz der Anwendung notwendigen Zustand repräsentieren, wurden mit \textit{Event Sourcing} persistiert. Dies sind folgende Actors: 
\begin{itemize}
    \item{\textit{OperateFlight}}
    \item{\textit{FlightPassangerList}}
    \item{\textit{ChargingCoordinator}}
    \item{\textit{FlightTicket}}
    \item{\textit{FlightNumber}}
    \item{\textit{DistributedUniqueNamingService}}
\end{itemize}
Die ersten fünft genannten Actors sind Teil der Komponente \textit{Domain Service} und präsentieren die Domäne der \textit{TyrolSky}. Der verbleibende Actor \textit{DistributedUniqueNamingService} dient zur Implementierung der unterschiedlichen \textit{Query}-Aufbereiter des \textit{Query-Services}, welche bereits in Abschnitt \ref{subsubsub:implementation:queryActorModel:resultPreparator} beschrieben wurden.

\subsection{Version Problematik}
Es muss sichergestellt werden, dass Events bei einer späteren Wiedereinspielung gleich behandelt werden als zu dem Zeitpunk, an welchem das Ereignis tatsächlich aufgetreten ist. Ändert sich die Implementierung für einen bestimmten Event-Typ innerhalb der Methode \textit{UpdateState()}, welche auch vergangene Events verarbeitet, kann es zu Inkonsistenzen der Anwendung kommen. Deshalb muss bei einer Anpassung von dem vorhandenem Code der \textit{UpdateState()} Methode entschieden werden, ob dies Codeänderung auch für ältere Events zulässig ist. Wird durch die Codeanpassung ein anderer Actor State als zuvor berechnet, kann nicht mehr der Originalzustand zum Zeitpunk des Ereignisses herbeigeführt werden. \\
Soll ein neues Verhalten für ein bereits vorhandenes Ereignis benötigt werden, wird ein neuer Event-Typ eingeführt. Durch die Versionierung von Event-Typen können keine Inkonsistenzen durch Codeänderungen in das System gebracht werden. 


\section{Verteilung der Anwendung}
\label{subsec:implementation:ApplicationDistribution}
Ein wesentliche Anforderung an die Praxisanwendung ist dem Anforderungskatalog unter Kapitel \ref{sec:Eruierung:technicalRequierements} entsprechend, die Fokussierung der Softwarearchitektur auf die auf Verteilung der gesamten Anwendung. Um dies zu erreichen, wurde bereits die Anwendung in verschiedene Services unterteilt. Diese wurden in Abschnitt \ref{sec:implementation:serviceAndComponentOrientation} näher beschrieben. \\
Für die Erstellung von Instanzen dieser Komponenten, wurden vier verschiedene Konsolenanwendungen implementiert. Wobei für jeden Service ein eigene Anwendung geschrieben wurde, mit Ausnahme vom \textit{Domain-Service} und \textit{Command-Service}, welche in einer gemeinsamen Anwendung operieren. \\
Die Verbindung zwischen den einzelnen Komponenten wird mit der Cluster Unterstützung, welche \textit{Akka.net} mitliefert und im nachfolgenden Abschnitt \ref{subsec:implementation:akka:cluster} beschrieben ist, umgesetzt. 

\subsection{Cluster}
\label{subsec:implementation:akka:cluster}
Die Cluster Funktionalität von \textit{Akka.net} bietet die Möglichkeit, zusammengehörende Anwendungen über ein Netzwerk zu verbinden und somit einen Cluster zu formen. Für den Benutzer der Anwendung harmoniert der Cluster als eine geschlossene Einheit. Dazu werden mittels einem \textit{Peer-to-Peer} Netzwerk alle teilnehmenden Anwendungen untereinander verbunden. Über ein Protokoll, das sogenannte \textit{Gossip} Protokoll, genauer beschrieben in Abschnitt \ref{subsec:implementation:gossip}, werden Informationen zum Zustand des Clusters ausgetauscht. Jeder am Cluster Teilnehmende Anwendung wird als Node bezeichnet. Somit bilden zwei oder mehr Nodes einen Cluster \citep{akkaInAction}. \\
Jede Anwendung bekommt Rollen zugewissen, welche es zur Laufzeit übernehmen kann. Eine Rolle definiert in \textit{Akka.net} ein Aufgabengebiet innerhalb des Clusters. Für \textit{TyrolSky} wurden die fünf bereits definierten Services in Abschnitt \ref{sec:implementation:serviceAndComponentOrientation} als Rollen herangenommen. Tritt eine Anwendung dem Cluster bei, so gibt sie während dem Eintrittverfahren bekannt, welche Rollen sie im Cluster übernehmen kann. Eine Rolle kann somit mehrfach im Cluster vorhanden und somit mehrfach redundant ausgelegt sein. \\
Die Kommunikation zwischen den Komponenten erfolgt innerhalb des \textit{Akka.net} Cluster entweder über Routing-Mechanismen, siehe Abschnitt \ref{subsec:implementation:akkaRouting} oder über verteilte Daten, welche in \textit{Shards} organisiert sind, siehe dazu Abschnitt \ref{subsec:implementation:akkaSharding}. 

\subsection{Routing}
\label{subsec:implementation:akkaRouting}
In Kapitel \ref{sec:actor:patterns:routing} wurde bereits darauf eingegangen, dass es in einem System Actors geben kann, welche für die Verteilung von Nachrichten an verschiedene Actors zuständig sind. Die Verteilung von Nachrichten zwischen den teilnehmenden Nodes, in einem auf \textit{Akka.net} Cluster aufbauenden System, wird durch eben solche Router bewerkstelligt \citep{Akka.netCommunityAkka.NETDocumentation}. \\
Das Framework bietet dabei bereits verschiedene Typen von Actors an, welche Nachrichten annehmen und an andere Actors auf entfernten Nodes weiterleiten. Diese, auf das weiterleiten von Nachrichten spezialisieren Actoren, werden Router genannt. Bei der Erstellung eines Routers wird über Parameter angegeben, welche Actoren als Ziel dienen können und welche Routingsstrategie verwendet werden soll. Weiters können die möglichen Rollen, auf welchem sich die Ziele befinden, eingeschränkt werden. Nachfolgend wird die verwendeten Routingsstrategie innerhalb von \textit{TyrolSky} beschrieben. \\
Sämtliche Anfragen an das System haben ihren Startpunkt bei der Komponente \textit{API-Service}. Von dieser müssen Anfragen an die betreffenden Komponenten weitergeleitet werden. Die entsprechende Komponente kann sich, da die Anwendung mit dem \textit{CQRS}-Prinzip arbeitet, nur in den Rollen \textit{Command-Service} und \textit{Query-Service} befinden. Bei einer ankommenden Anfrage kann der \textit{API-Service} aufgrund des Types der Anfrage entscheiden, ob die weitere Verarbeitung in einem \textit{Query} oder \textit{Command-Service} stattfinden soll. Dementsprechend wird die Anfrage an den Router, welcher für den entsprechenden Service zuständig ist, weitergeleitet. Der Router überwacht mithilfe der Informationen welche er über \textit{Gossip} erhält, den Status von anderen Nodes mit der entsprechenden Rolle im Cluster. Empfängt der Router eine Nachricht, so kann er einen der in Frage kommenden, verfügbaren Nodes auswählen, und die Nachricht an diesen zustellen. \\
Bei der Auswahl des entsprechenden Zielnodes wird dabei für diesen Router das \textit{Round-Robin} Verfahren eingesetzt. Somit ist eine gleichmäßige Verteilung der Nachrichten auf die unterschiedliche Host möglich. Es gibt neben dem \textit{Round-Robin} Routingverfahren noch andere bereits implementierte Routingsstrategien innerhalb von \text{Akka.net}. Unter anderem werden die Strategien Zufälliges Routing, Hashingrouting oder kleineste Mailbox vom Framework mitgeliefert \citep{Akka.netCommunityAkka.NETDocumentation}. Jedoch wurden diese im zuge der Umsetzung von \textit{TyrolSky} nicht verwendet. \\
In Abbildung \ref{fig:implementation:routing} ist das Zusammenspiel der drei Komponenten \textit{API-Service}, \textit{Query-Service} und \textit{Command-Service} zu sehen, wobei von den zwei zuletzt genanten jeweils zwei Instanzen am Cluster teilnehmen. Die Router auf dem Node \textit{A} leiten die Nachrichten abwechselnd an die zwei Nodes \textit{B} und \textit{C} oder an \textit{D} und \textit{E} weiter. 

\begin{figure}
    \centering
    \includegraphics[width=\linewidth]{gfx/implementation/ClusterRouter}
    \caption{Ein verteiler \textit{Round-Robin} Router welcher Nachrichten von der Komponente \textit{API} auf andere Komponenten verteilt.}
    \label{fig:implementation:routing}
\end{figure} 

\subsection{Gossip}
\label{subsec:implementation:gossip}
Bereits in Abschnitt \ref{subsec:implementation:lighthouse} wurde die Komponente \textit{Lighthouse} vorgestellt, welche als Einstiegspunkt für alle teilnehmenden Nodes verwendet wird. Die Verbindung zu den restlichen, im Cluster vorhandenen Nodes, wird über das \textit{Gossip}-Protokoll hergestellt. \\
Der Name \textit{Gossip} leitet sich vom Geschwätz mehrerer Personen ab, jeder erzählt einem anderen etwas, und so wissen alle über jeden anderen  bescheid \citep{Akka.netCommunityAkka.NETDocumentation}. Das gleiche Prinzip wird in diesem Protokoll verwendet, um Änderungen an einem einzelnen Node allen anderen Nodes im Cluster bekannt zu geben. \\
Nodes mit der Rolle \textit{Lighthouse}, sind der Start dieser Kommunikation. Deshalb müssen diese Nodes auch über eine fixe Adresse verfügen, im Fall von \textit{TyrolSky} sind das Ip-Adresse sowie Port. Komponenten welche nun dem Cluster beitreten wollen, melden sich bei einem der \textit{Lighthouse}-Komponenten an, und geben bekannt, über welche Adresse sie erreichbar sind. Über das \textit{Gossip}-Protokoll wird nun jedem bereits im System beigetretenen Node mitgeteilt, dass sich ein neuer Node im Cluster befindet. \\
In der Kommunikation zwischen den Nodes, wird jedem Node auch mitgeteilt, welche Rollen eine teilnehme Instanz besitzt. Dies wird dann anschließend von den Routern, siehe Abschnitt \ref{fig:implementation:routing}, verwendet, um Nachrichten an diese Rolle zuzustellen. Weiters beinhaltet das Protokoll auch Informationen über den aktuellen Zustand der einzelnen Nodes. So werden einer fehlerhaften Verbindung zwischen zwei Nodes, alle anderen Nodes im Cluster über den Fehlerzustand benachrichtigt. Ist ein Node nicht mehr erreichbar, ohne das eine kontrollierte abmeldung vom Cluster stattgefunden hat, kann er jedoch nicht einfach aus dem Cluster entfernt werden. Um den fehlerhaften Zustand des Clusters zu lössen, sollte eine Lösungsstrategie implementiert werden welche den Cluster wieder in eine fehlerfreien Zustand führt, siehe dazu den nachfolgenden Abschnitt \ref{subsec:implementation:splitBrain}. \\ 
Das \textit{Gossip}-Protokoll dient neben der Zusammenfügung der einzelnen Node zu einem Cluster auch zum Austausch von Informationen über den aktuellen Zustand der einzelnen Nodes. Muss vom Protokoll eine Entscheidung getroffen werde, wie beispielsweiße die Entscheidungen, wie beispielseise ob ein neuer Node dem Cluster beitreten darf, benötigen einen Node, welcher diese Entscheidungen treffen kann. Dazu wird ein Node ausgewählt, welcher die entsprechende Führung über den Cluster übernimmt. Im Fall von \textit{TyrolSky} ist das immer der ältester Node im Cluster. Wird dieser vom Cluster entfernt, übernimmt die Führung der nächste älteste im Cluster enthaltende Node. Über diesen, in \textit{Akka.net} bezeichneten \textit{Leader}, werden entscheidungen getroffen welche für \textit{Gossip} relevant sind \citep{akkaInAction}.  

\subsection{Split-Brain} 
\label{subsec:implementation:splitBrain}
Wie bereits in Kapitel \ref{sec:distributedSystems:capTheorem} besprochen, kann bei einer Verbindung zwischen zwei Geräten, jederzeit ein Problem auftreten welche die Kommunikation zwischen den Teilnehmern stört oder gänzlich unterbricht. Bei einer solchen Störung, ist es dem betroffenen Node nicht mehr möglich, am \textit{Gossip} teilzunehmen und seinen Zustand anderen mitzuteilen. Weiters wissen anderen Cluster Teilnehmer den Grund für Trennung der Kommunikation nicht. Deshalb können diese auch keine Aussage treffen, ob die Trennung des Teilnehmers nur temporär ist oder ob der Partner gar nicht mehr existiert. \\
Ein ähnliches Szenario kann auch auftreten, wenn zwei oder mehr Gruppen von Nodes zwar untereinander eine aufrechte Kommunikation führen können, jedoch die Gruppen selbst vomeinander getrennt wurden. So agiert jede Gruppe des Clusters für sich selber und kann sich  mit den anderen Gruppen nicht mehr verständigen. Würde nun jeder Node im Cluster andereren Nodes, welche für ihn nicht erreichbar sind, selbstständig entfernen, so würde jede Gruppe einen eigenen, neuen Cluster bilden, was als \textit{Split-Brain} bezeichnet wird \citep{networkIsReliable}. Bildet sich aus einem Cluster, durch gegenseitiges Entfernen, mehrere neue Clusters, so bilden sich zwei oder mehrere Parallelstrukturen welche die Konsistenz der gesamten Applikation verletzen können. Unter anderem werden die Prinzipien von \textit{Cluster Singletons} oder \textit{Sharding} verletzt und dies kann entsprechend zu redundanzen führen. \\
In \textit{Akka.net} gibt es bereits einige Strategieren welche für das \textit{Split-Brain} Problem angewandt wurden. Jede Strategie hat jedoch einen Nachteil welcher teilweise sogar zum Stillstand der gesamten Anwendung führen kann. Nachfolgend die laut \cite{Akka.netCommunityAkka.NETDocumentation} bereits verfügbaren Strategien im Framework.
\begin{itemize}
    \litem{Statische Mehrheit}
    Jedem teilnehmenden Node wird die gleiche statische Nummer beim Start übergeben. Kann ein Node weniger Teilnehmer erreichen als die Nummer angibt, so beendet sich der Node selbstständig. Dadurch werden kleine gesplittete Gruppen automatisch beendet. Teilt sich jedoch das Netzwerk in mehr als zwei Gruppen auf, so werden alle Nodes im gesamten Netzwerk beendet. 
    \litem{Behalte Mehrheit}
    Wird der Cluster durch ein Netzwerkproblem geteilt, wird anhand der zuletzt verfügbaren Informationen über den gesamten Cluster analysiert, ob der nun erreichbare Teil des Clusters, die Mehrheit der vor der Unterbrechung  beteiligten Nodes  erreichen kann. Entspricht der neue Cluster der Mehrheit, so werden die Nodes nicht beendet. Ansonsten werden alle Nodes des gesplitteten Clusters beendet. Auch hier kann das Aufteilen des Clusters in mehr als zwei Gruppen zum Stillstand des gesamten Clusters führen.
    \litem{Behalte Älteste}
    Durch das \textit{Gossip Protokoll}, siehe Abschnitt \ref{subsec:implementation:gossip}, wird  mitgeteilt zu welchem Zeitpunkt ein Node dem Cluster beigetretten ist. Darauf basierend, kann berechnet werden welcher Node der älteste im Cluster ist. Sobald sich der Cluster aufteilt, werden alle Nodes beendet, welche den ältesten Node nicht erreichen können. Dadurch ist sichergestellt, das auch bei einer mehrfachen aufteilung nicht der ganze Cluster beendet wird. Wird jedoch der teil vom Node getrennt welcher den ältesten Nodes und einige wenige andere Cluster beinhalteten, so wird der gesamte größere Teil des Clusters beendet. Die Anwendung ist zwar nicht beendet, jedoch wurden mehr Nodes beendet als Nötig. 

    \litem{Behalte Referenz}   
    Es wird ein fixer Nodes bestimmt, welcher für jeden Node erreichbar sein muss. Erreicht ein Node diesen nicht mehr, so beendet sich dieser automatisch. Dies führt dazu, dass wenn der Referenzierte Node im gesamten Cluster nicht mehr erreichbar ist, die gesamte Anwendung beendet wird. Jedoch ist eine Aufteilung in zwei Teilen, der sogenannte \textit{Split-Brain} nicht möglich.
\end{itemize}
Für die Implementierung der \textit{TyrolSky} Anwendung wurde die Strategie \textit{Behalte Älteste} gewählt. Eine Zersplitterung des Clusters ist somit äußerst unwahrscheinlich. Das Risiko zu viele Nodes im Fehlerzustand zu beenden und das System somit kurzfristig zu verlangsamen wird eingegangen, um die Datenkonsistenz für das \textit{Sharding} nicht zu verletzen.   

\subsection{Globale Dienste}
\label{subsec:implementation:singeltons}
Bereits in Abschnitt \ref{subsubsub:implementation:queryActorModel:resultPreparator} wurde ein Actor benötigt, welcher um korrekt zu funktionieren nur exakt einmal im gesamten Cluster instanziiert werden darf. Im konkreten Beispiel war dies erforderlich, da der Actor für andere Actoren eindeutige, wiederverwendbare Namen generieren muss. In \textit{Akka.net} werden solche Actoren als \textit{Singletons} bezeichnet. Das Framework gewährleistet, dass die Actors welche als Singletons deklariert sind, nur einmal im Cluster vorhanden sind. Dabei wird auch berücksichtigt, dass wenn der Node auf welchem der Singleton Actor läuft, vom Cluster entfernt wird, der Singleton-Actor auf einen anderen Node im Cluster verschoben wird. Dadurch ist der Actor, mit Ausnahme der Zeit welche Prozess des verschiebens benötigt, zu jederzeit im System exakt einmal verfügbar. \\
Für die Überwachung und das Management der einzelnen Singleton-Actors ist der führende Cluster Node zuständig, siehe dazu Abschnitt \ref{subsec:implementation:gossip}. Jedem Singleton kann auch eine Cluster Rolle zugewissen werden, anschließend werden diese Singletons nur auf Nodes, welche dieser Rolle zugeordnet sind, ausgerollt. \\
Für die Implementierung von \textit{TyrolSky} wurden zwei Singletons eingesetzt. Beide werden für die Umsetzung der Namenverwaltung für die beiden Ergebnissaufbereiter im \textit{Query Service} benötigt.  Denn zwei Singletons wurde die Rolle \textit{Query} zugewiesen,  womit sie auch nur auf den Nodes laufen, auf welchem sich die Ergebnissaufbereiter befinden.

\subsection{Verteilung transaktionaler Daten}
\label{subsec:implementation:akkaSharding}
Ein wesentlicher Teil der Komponente \textit{Domain Service}, Abschnitt \ref{subsec:implementation:domainService}, ist das Verteilen der Entitäten auf verschiedene Nodes der Rolle \textit{Domain Service}. Wie im Abschnitt über den \textit{Domain Service} bereits erwähnt, wird die Verteilung der Entitäten über den gesamten Cluster mit dem Prinzip des \textit{Shardings} umgesetzt. \\
Die Idee hinter \textit{Sharding} basiert auf Vorgehensweisen für verteilte Datenbanken und bedeutet laut \cite{shardingCattell}, dass Daten mithilfe eines Schlüssels auf unterschiedliche Hosts verteilt werden. Dabei bekommt jeder beteiligte Host einen bestimmten Bereich der Schlüsselmenge zugeteilt. Alle Hosts besitzen die Informationen welche Schlüsselbereiche zu welchem Host zugeordnet sind und können davon ausgehen, dass der dazugehörigen Datensatz sich an dem entsprechend Host befindet. Dadurch ist eine effektive Verteilung von Daten auf unterschiedliche Hosts möglich, ohne das dabei jeder beteiligte Host sämtliche Daten besitzen muss. Jedoch muss für eine Abfrage immer der dazugehörige Schlüssel bekannt sein. \\
Die Implementierung von \textit{TyrolSky} benützt diese Technik mithilfe von \textit{Akka.net} um Actoren über mehrere Nodes zu verteilen und dabei sicherstellen, dass diese von jedem anderen Node erreicht werden können. Dazu werden die Entitäten, also die Actoren welche im Sharding Cluster leben sollen, in \textit{Shards} unterteilt. Auf jedem Node welcher \textit{Shards} übernehmen soll, wird ein oder mehrere  \textit{Shard Regions} gestartet. Dieser übernimmt, wie in Abbildung \ref{fig:implementation:actorSharding} dargestellt, die im zugewiesenen \textit{Shards} und startet die darin befindlichen Entitäten. Die Anzahl an möglichen \text{Shards} wird darbei für eine Anwendung fixiert. Im Falle von \textit{TyrolSky} werden {100} \textit{Shards} verwendet. Diese werden gleichmäßig an die verfügbaren \textit{Shard Regions} verteilt. Wird eine neue Region hinzugefügt, erfolgt eine Umverteilung der \textit{Shards}, womit auch die darin befindlichen Actors umverteilt werden. 

\begin{figure}
    \centering
    \includegraphics[width=0.8\linewidth]{gfx/implementation/Sharding}
    \caption{Verteilung von Entitäten in Shards, welche selber wieder in \textit{Shard Groups} organisiert sind. Die Gruppen können zwischen den Nodes verteilt werden. }
    \label{fig:implementation:actorSharding}
\end{figure} 

\subsubsection{Nachrichten in \textit{Shards}}
Ein Grundprinzip von \textit{Sharding} ist die Unterteilung von Entitäten in unterschiedliche Gruppen, sogenannte \textit{Shards}. Dies wird in der \textit{Akka.net} Cluster Implementierung, über die Zustellung von Nachrichten gelöst. Dazu werden Nachrichten nicht wie bei Actor Systemen direkt an den Actor gesendet, sondern zuerst an einen Sharding Manager. Diese Nachrichten enthalten eine Identifikationsnummer der Entität, an welche sie eigentlich gerichtet sind. Der Manager interpretiert diese Nachricht und errechnet sich aus der Identifikationsnummer eine \textit{Shard} Identifikation. Anschließend prüft er in welchem \textit{Shard Group} sich der \textit{Shard} befindet und auf welcher Node für diese Gruppe verantwortlich ist. Anschließend sendet er die Nachricht an den \textit{Shard Manager} auf dem entsprechenden Node. Hat sich dazwischen der \textit{Shard} auf einen anderen Node verschoben, kann der Node die Prüfung erneut durchführen und die Nachrichten an den neuen Zielort weiterleiten. Ist jedoch die gewünschte Entität am Node enthalten, stellt der dortige \textit{Shard Manger} die Nachricht an den entsprechenden Actor zu. \\
Für die Berechnung der \textit{Shard}-Identifikation auf Basis der Entitäts-Identifikation, wird in der Implementierung auf eine Hashfunktion zurückgegriffen. Der \textit{Shard Manager}, welcher die Nachricht an die zuständige \textit{Shard Region} weiterleiten soll, berechnet einen Integer Hashwert von der Entitäts-Identifikation. Diesen Wert Modulo der Anzahl an möglichen \textit{Shard Regions} ergibt den \textit{Shard}, welcher für diese Entität zuständig ist. Die Anzahl der möglichen \textit{Shards} darf sich demnach zur Zaufzeit des gesamten Clusters nicht verändern. Ansonsten würde durch die beschriebene Berechnungsfunktion  neue \textit{Shards} errechnet werden und die Verteilung der Nachrichten schlägt fehlt. \\
Die Erzeugung von neuen Actoren in einem Shard wird durch Nachrichten an die noch nicht vorhandenen Actoren gestartet. Erhält ein \textit{Shard Manager} eine Nachricht an eine Entität welche einem seiner \textit{Shards} zugeordnet werden kann und ist diese Entität noch nicht gestartet, so instanziiert der \textit{Shard Manager} einen neuen Actor. Dieser bekommt nun die Entitäts-Identifikation von der Nachricht zugewissen.


\section{Garantierte Nachrichtenübermittelung}
Die gesamte Kommunikation zwischen Actors baut auf Nachrichten zwischen den beteiligten Actors auf. Wie bereits in Abschnitt \ref{sec:actor:patterns:guaranteedDelivery} diskutiert, kann die Zustellung einer Nachricht nicht, ohne verhältnismäßig viel Aufwand, garantiert werden. Wird in der \textit{TyrolSky} Anwendung eine Nachricht zwischen zwei Actors ausgetauscht wird die Nachricht höchstens einmal zugestellt. Das bedeutet, dass eine Nachricht unter bestimmten Umständen, wie beispielsweise Netzwerkfehler, nicht zugestellt wird und dies der Versender ohne spezielle Erkennungsmuster nicht mitbekommt. \\
Das Anwendungsumfeld von \textit{TyrolSky} benötigt jedoch auch Kommunikation, wo sichergestellt wird, dass die Zustellung einer Nachricht tatsächlich bis zum Empfänger funktioniert. Dies ist unter anderem der Fall, wenn ein Ticket sich selber storniert. Dabei muss sichergestellt sein, dass die zugehörige Passagierliste über die stornierung Informiert wird, um anschließend den dazugehörigen Sitzplatz für andere wieder freizugegeben. Dafür wurde die Garantierte Nachrichtenzustellung implementiert. Diese Zustellvariante ist eine Implementierung der in \ref{sec:actor:patterns:guaranteedDelivery} vorgestellten Zustellvariante \textit{garantierte Zustellung mindestens einmal}. \\
Die Ausführung der Garantierten Nachrichtenzustellung basiert auf der Umsetzung des \textit{Event-Sourcings}, beschrieben unter \ref{sec:implementation:eventSouring}. Dabei wird anstelle der Nachrichtenübermittelung an einen Actor mit den Methoden \textit{Tell} oder \textit{Ask}, siehe Abschnitt \ref{subsec:implementation:akka:cluster}, ein Event beim Sender Actor erzeugt, welches die Zustellung einer Nachricht an einen Actor signalisiert. Nach der erfolgreichen Speicherung des Events mit dem Namen \textit{GuaranteedMessageSent}, wird die Nachricht in einer Wrapper-Nachricht vom Typ \textit{GuranteedMessage} zum Empfänger zugstellt. Zusätzlich wird im \textit{State} des Sender Actors abgespeichert, dass diese Nachricht versendet wurde und vom Sender eine Bestätigung erwartet wird. Trifft innerhalb einer konfigurierbaren Zeit, im Falle von \textit{TyrolSky} sind dies 5 Sekunden, keine Bestätigung für diese Nachricht ein, so wird diese erneut an den Empfänger zugestellt. Durch die Implementierung der Zustellvariante mittels \textit{Event Sourcing}, werden nicht bestätigte Nachrichten auch nach einem Neustart des Actors wieder versandt. \\
Trifft eine Nachricht vom Typ \textit{GuaranteedMessage} ein und wird vom Actor bearbeitet, so wird eine Bestätigungsnachricht vom Typ \textit{Confirm}, welcher auch Informationen über die Original-Nachricht enthält, an den Sender der ursprünglichen Nachricht gesendet. Empfängt dieser nun die Nachricht vom Typ \textit{Confirm}, so markiert er die Original-Nachricht als bestätigt. Somit kann der Sender bestätigen, dass die Nachricht an seinem Zielort angekommen ist. Ist eine Zustellung nach mehreren versuchen trotzdem nicht möglich, kann der Sender der Nachricht entsprechend darauf reagieren, hierfür ist jed  och eine angepasste Implementierung je nach Anwendungsfall notwendig. \\

\section{Externe Schnittstelle}
\label{sec:implementation:externalApi}
Neben der Interaktion mit Benutzern der \textit{TyrolSky}-Anwendung über die \textit{API}-Komponente, wird auch eine Kommunikation mit einem Fremdsystem benötigt. Um Abbuchungen an einem Konto vorzunehmen, wird, wie im Anforderungskatalog unter Abschnitt \ref{sec:Eruierung:sec:Eruierung:technicalRequierements} gefordert, eine externe Anwendung angebunden. Diese Bankenanwendung operiert als eigenständige Anwendung, und simuliert eine Bankenschnittstelle um Kontoabbuchungen von Kunden vorzunehmen. Die Simulationsanwendung selbst ist im nachfolgenden Kapitel \ref{subsec:implementation:bankApi} genauer beschrieben. \\
Der bereits in Abschnitt \ref{subsub:implementation:ChargingCoordinator}  beschriebene Buchungskoordinator verteilt Abbuchungstransaktionen mithilfe von \textit{Sharding} innerhalb des Clusters auf verschiedene Nodes. Jede Transaktion benötigt jedoch eine Kommunikation mit der Bankenschnittstelle, um die Transaktion durchzuführen und sie mit dem Bankensystem abzugleichen.  \\
Dazu wird auf jedem Node welcher Buchungen durchführen kann, das sind alle Nodes mit der Rolle \textit{Domain-Service}, ein \textit{BankingActors} Actor gestartet, welcher für die Kommunikation mit dem Bankenanwendung zuständig ist.
 Möchte nun eine Transaktion eine Kommunikation mit der Bankenanwendung durchführen, so sendet den Befehl zu Kommunikation in Form einer Nachricht an den \textit{BankingActor} welcher sich auf dem gleichen Host befindet wie der Transaktions-Actor. Wird während der Kommunikation zur Bank, der betroffene Transaktions-Actor durch \textit{Sharding} verschoben, betrifft dies nicht den BankingActor selbst, da dieser fix auf jedem Host als eigene Actor Instanz vorhanden ist. 

\subsection{Banken API}
Für die Anbindung einer externen Bank wurde eine eigene Software geschrieben, welche einen einfache Banken Schnittstelle zur verfügung stellt. Die \textit{BankChargingAPI} ist komplett getrennt von der restlichen Implementierung der \textit{TyrolSky} Anwendung und wird über eine eigene \textit{REST} Schnittstelle angesprochen. \\
Die Kernaufgabe dieser Bankensimulation ist es, Bankkonnten zu verwalten. Dazu werden in einer relationalen Datenbank Kontoinformationen abgebildet. Dabei hat jedes Konto einen Namen sowie ein aktueller Guthabenstatus. Für die Bankentransaktionen, nicht zu verwechseln mit den Transaktion innerhalb der \textit{TyrolSky}-Anwendung, wird eine eine eigene Tabelle geführt welche Informationen über die Transaktion führt. Jede einkommende Anfrage für eine Kontobewegung wird in einer Transaktion abgebildet. Diese enthält Informationen über den Status der Transaktion sowie deren Transaktionsbetrag. Über die Schnittstelle kann auch der Status einer Transaktion abgefragt werden. Diese Transaktionsstatus Abfrage wird auch von der \textit{TyrolSky}-Anwendung verwendet um die eigene Flugbuchung zu kontrollieren, beschrieben unter Abschnitt \ref{subsub:implementation:ChargingCoordinator}. \\
Für die Implementierung der Simulationsanwendung wurde auf eine \textit{SQL-Datenbank} zurückgegriffen. Jeder Zugriff auf die Datenbank erfolgt in einer Transaktion mit dem strengen Transaktionslevel \textit{Serializable}, wodurch keine Inkonsistenzen der Kontoinformationen entstehen können. Weiters ist durch die Transaktionstabelle eine Nachvollziehbarkeit der Transaktionen möglich. \\
Die Simulationsanwendung \textit{BankChargingAPI} bietet folgende Schnittstellen zur an:
\begin{itemize}
    \litem{ChargeAccount}
        Mit Angabe des gewünschten Kontos sowie des zu abbuchenden Betrages, wird eine Transaktion gestartet. Für den Aufruf, wird weiters eine Transaktionsnummer angegeben, mit welcher später der Status der Transaktion abgefragt werden kann.
    \litem{GetTransactionStatus}
        Durch angabe der Transaktionsnummer wird der aktuelle Status der Transaktion zurückgegeben. 
\end{itemize}
Durch die Aufteilung in zwei Methoden, ist es möglich im Fehlerfall Transaktionsinformationen nicht zu verlieren. Wird beim Aufruf der \textit{ChargeAccount} Schnittstelle kein Ergebnis zurückgeliefert so kann der Status trotzdem über \textit{GetTransactionStatus} zu einem späteren Zeitpunkt abgefragt werden.

\label{subsec:implementation:bankApi}

\section{Testapplikation}
\label{subsec:implementation:TestApplikation} 

\section{Verwendete Umgebung}
Für die Umsetzung sowie den Betrieb der \textit{TyrolSky} Anwendung wurden unterschiedliche Fremdkomponenten eingesetzt. Die Tabelle \ref{tab:implementation:EnvironmentVersions} gibt einen Überblick über die maßgeblich eingesetzten Komponenten sowie Softwaresystemen. 

\begin{table}[h]
  \centering
  \begin{tabular}{llp{6.5cm}}
  Name      & Version    & Beschreibung \\ \hline
  C\#     & 7.0        & Programmiersprache mit welcher alle Teile der Anwendung umgesetzt wurden. \\
  DotNetCore   & 2.0.0       & Laufzeitumgebung der Entwickelten C\# Programme \\
  Akka.net    & 1.3.5       & Framework für die Umsetzung des  Actor-Models mit C\# \\
  Akka.net Sharding & 1.3.5-beta60  & Erweiterung für Akka.net um Sharding zu ermöglichen \\
  Akka.net Persistence Sql & 1.3.2  & Ermöglicht die Speicherung des Akka.net EventStores in einer SQL Datenbank\\
  SqlServer    & 2017 14.0.3025.34 & Datenbank Server für den EventStore sowie die Datenbank für die Testanwendung\\
  Docker     & 18.03.1      & Containerbasierte Infrastrukturumgebung für die benötigten verschiedenen Anwendungen\\
    \end{tabular}
    \caption{Die in \textit{TyrolSky} verwendeten Bibliotheken sowie die für den Testbetrieb verwendete Software}
    \label{tab:implementation:EnvironmentVersions}
    \end{table}