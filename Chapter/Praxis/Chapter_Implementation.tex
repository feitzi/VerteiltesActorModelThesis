\chapter{Umsetzung eines Verteilen Transaktionalen Systems mit dem Actor Model} 
\label{cha:practicalDevelopment}

Während der Umsetzung des Anforderungskataloges aus Abschnitt \ref{sec:Eruierung:technicalRequierements} wurde versucht das erarbeitete Wissen aus den theoretischen Kapitel praktisch anzuwenden. Die implementierte Architektur des Flugbuchungssystem legt seinen Schwerpunkt auf die Verteilung des gesamten Systems. Mit der Nachfolgend vorgestellten Implementierung ist es möglich Anzahl an Instanzen der verschiedenen Komponenten Während des Betriebes beliebig anzupassen. \\
Das folgende Kapitel erklärt den Aufbau der praktischen Anwendung. Nach der Einführung des verwendeten Frameworks um mit dem Actor Model arbeiten zu können, wird zuerst der Komponentenorientierte Aufbau der Implementierung vorgestellt. Anschließend wird auf die Implementierung der Abfrage und Befehlverarbeitung eingegangen, welche auf die starke Orientierung auf Verteilung angepasst ist. Nachfolgend wird die Implementierung der Garantierten Nachrichtenzustellung aus Abschnitt \ref{sec:actor:patterns:guaranteedDelivery} vorgestellt. Abschließend wird das Kapitel die Testapplikation vorstellen mit welcher die verteilte Anwendung geprüft wurde sowie auf die während der Entwicklung verwendete Environment Umgebung einehen.

\section{Einführung in das \textit{Akka.net} Framework}
Das Framework \textit{Akka} stellt das Actor-Model für \textit{Java} und \textit{Scala} Entwickler zur verfügung. Bei Entwickler dieser Sprachen konnte sich das Framework, und somit auch das Actor-Model einen guten Namen machen. Für \textit{.Net} Entwickler mit den Sprachen \textit{C\#} und \textit{F\#} wurde das Projekt mit Hilfe der OpenSource Community unter dem Name \textit{Akka.Net} für die \textit{.Net} Plattform portiert. Somit sind die funktionalität und der Aufbau der Frameworks zwischen \textit{Akka} und \textit{Akka.net} identisch, wenngleich auch die Implementierung, aufgrund der Sprach und Laufzeitunterschiede zwischen der \textit{Java Virtual Machine (JVM)} und der \textit{Common Language Runtime (CLR)}, siehe hierzu \cite{JvmVsClrsinger2003jvm}, unterschiedlich ist. \\
Die Implementierung der hier beschriebenen Anwendung erfolgt für die \textit{CLR}, und somit mit Hilfe des Frameworks \textit{Akka.Net}. 

\subsection{Ein Actor in \textit{Akka.Net}}
Wie bereits erwähnt wird eine Anwendung auf Basis von \textit{Akka.Net} mit der Objektorientierten Sprache \textit{C\#} geschrieben. Somit werden auch Actors als Klassen repräsentiert. In \textit{Akka.Net} gibt dafür Basisklassen welche die Funktionalität eines Actors, wie die Nachrichtenübermittelung oder das Verhalten auf empfangene Nachrichten, beinhalten. Die Actor Klasse selbst bietet jedoch für andere Klassen keine eigenen Methoden zur verfügung. Somit ist der Zugriff von Außen, wie in Abschnitt \ref{actor:requirements:shareNothing} beschrieben, nicht möglich. \\

\subsubsection{Nachrichten}
Da ein Actor selbst keine Methoden anbietet erfolgt die Interaktion mit der Umwelt des Actors über Nachrichten. Dies wird in \textit{Akka.Net} durch Objekte sichergestellt. Eine Nachricht selbst muss in \textit{Akka.Net} keine bestimmte Struktur aufweisen, kann somit von jedem beliebigen Datentyp sein. Um jedoch die zustellung von Nachrichten über verschiedene Hosts zu gewähreisten, siehe dazu Abschnitt \ref{sub:sec:implementation:ApplicationDistribution}, sollte eine Nachricht serialisierebar sein. 

\subsubsection{Erstellung}


 

\section{Service Architektur}

\subsection{\textit{Lighthouse}}

\subsection{\textit{REST}-Schnittstelle}

\subsection{Abfragen}

\subsection{Domain-Service}

\section{Implementierung von Event Sourcing}

\section{Verteilung der Anwendung}
\label{sub:sec:implementation:ApplicationDistribution}

\subsection{Fixe Dienste}

\subsection{Verteilung transaktionaler Daten}
Erklärung von Sharding

\section{Garantierte Nachrichtenübermittelung}

\section{Externe Schnittstellen}

\section{Testapplikation}

\section{Verwendete Umgebung}
Welche Version, wie genau etc...
