\subsection{Domain-Service}
\label{subsec:implementation:domainService} 
Die gesamte Geschäftslogik wie das verhalten von Flügen, Tickets und anderen Entitäten wird in der Komponente \textit{Domain-Service}  abgebildet. Die Actoren welche sich in dieser Komponente befinden, wurden so abgebildet das die Repräsentation der Anwendung selbst keine Rolle spielt. Das bedeutet, dass die Implementierung in dieser Komponente unabhängig davon umgesetzt wurde, ob die Anwendung selbst über das \textit{CQRS} Prinzip umgesetzt wird und ob die Interaktion mit dem Benutzer über eine \textit{HTTP} Schnittstelle umgesetzt wurde. \\
Die hier repräsentierten Entitäten, welche selbst ja wieder als Actoren repräsentiert werden, beinhalten neben der eigenen Logik auch einen Zustand welcher Persistiert wird. Dies wird durch \textit{Event Soucing} ermöglicht, welches in Abschnitt \ref{imp:subsec:implementation:eventSouring} genauer beschrieben wird. Ereignisse welche in der Entität selbst auftretten, werden persistiert und bei einem späteren Neustart des dazugehörigen Actors wieder eingespielt. Mit diesem Prinzip sind alle nachfolgend beschriebenen Entitäten ausgestattet. \\
Um eine Verteilung der Entitäten zu ermöglichen werden diese in Form von \textit{Sharding} auf unterschiedliche Hosts verteilt. \textit{Sharding} selbst wird in Kapitel \ref{subsec:implementation:ApplicationDistribution} genauer betrachtet. Für eine Verteilung mittels \textit{Sharding} müssen Actoren bestimmt werden, welche anschließend verteilt werden. Alle Kindelement dieser Actoren werden anschließend an den gleichen Ort wie ihre Parents verschoben. Innerhalb des \textit{Domain Service} gibt es folgende Actoren welche Verteilt werden:
\begin{itemize}
    \item Flug Nummer (\textit{FlightNumber})
    \item Flug Ticket (\textit{FlightTicket})
    \item Abbuchungs Koordination (\textit{ChargingCoordinator})
\end{itemize}
Auf die genaue Aufgabe sowie die Implementierung dieser drei Actoren wird nachfolgen eingegangen.