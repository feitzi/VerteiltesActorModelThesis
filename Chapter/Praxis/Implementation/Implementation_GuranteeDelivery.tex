\section{Garantierte Nachrichtenübermittelung}
Die gesamte Kommunikation zwischen Actors baut auf Nachrichten auf. Wie bereits in Abschnitt \ref{sec:actor:patterns:guaranteedDelivery} diskutiert, kann die Zustellung einer Nachricht nicht garantiert werden, ohne verhältnismäßig viel Aufwand zu betreiben. Wird in der \textit{TyrolSky}-Anwendung eine Nachricht zwischen zwei Actors ausgetauscht, kann die Nachricht höchstens einmal zugestellt werden. Das bedeutet, dass eine Nachricht unter bestimmten Umständen, wie beispielsweise bei einem Netzwerkfehler, nicht zugestellt wird und dies der Versender, ohne spezielle Erkennungsmuster, nicht mitbekommt. \\
Das Anwendungsumfeld von \textit{TyrolSky} benötigt Kommunikation, bei der sichergestellt werden kann, dass die Zustellung einer Nachricht tatsächlich bis zum Empfänger funktioniert. Dies ist der Fall, wenn ein Ticket sich selbst storniert. Dabei muss garantiert werden, dass die zugehörige Passagierliste über die Stornierung informiert wird, um anschließend den dazugehörigen Sitzplatz für andere wieder frei zugegeben. Dazu wurde die garantierte Nachrichtenzustellung implementiert. Diese Zustellvariante ist eine Implementierung, der in \ref{sec:actor:patterns:guaranteedDelivery} vorgestellten Zustellvariante \textit{garantierte Zustellung mindestens einmal}. \\
Die Ausführung der garantierten Nachrichtenzustellung basiert auf der Umsetzung des \textit{Event-Sourcings}, beschrieben unter \ref{sec:implementation:eventSouring}.
Anstelle der Nachrichtenübermittelung an einen Actor mit den Methoden \textit{Tell} oder \textit{Ask}, siehe Abschnitt \ref{subsec:implementation:akkaMessaging}, wird ein Event beim sendenden Actor erzeugt. Dieses Event signalisiert eine gewünschte Nachrichtenzustellung an einen anderen Actor. Nach der erfolgreichen Speicherung des Events mit dem Namen \textit{GuaranteedMessageSent} wird die Nachricht in einer Wrapper-Nachricht vom Typ \textit{GuranteedMessage} zum Empfänger zugestellt. Zusätzlich ist im \textit{State} des Sender-Actors abgespeichert, dass diese Nachricht versendet wurde und der Actor vom Sender eine Quittierung der Nachricht erwartet. Trifft innerhalb einer konfigurierbaren Zeit, im Falle von \textit{TyrolSky} sind dies 5 Sekunden, keine Bestätigung für diese Nachricht ein, wird diese erneut an den Empfänger zugestellt. Durch die Implementierung der Zustellvariante mittels \textit{Event-Sourcing} werden nicht quittierte Nachrichten auch nach einem Neustart des Actors wieder versandt. \\
Trifft eine Nachricht vom Typ \textit{GuaranteedMessage} ein und wird bearbeitet, soll eine Bestätigungsnachricht vom Typ \textit{Confirm}, welcher auch Informationen über die Original-Nachricht enthält, an den Sender der ursprünglichen Nachricht gesendet werden. Empfängt dieser nun die Nachricht vom Typ \textit{Confirm}, markiert er die Original-Nachricht als bestätigt. Somit kann der Sender quittieren, dass die Nachricht an seinem Zielort angekommen ist. Ist eine Zustellung nach mehreren Versuchen trotzdem nicht möglich, kann der Sender der Nachricht entsprechend darauf reagieren, hierfür ist eine angepasste Implementierung je nach Anwendungsfall notwendig. \\
