\section{Garantierte Nachrichtenübermittelung}
Die gesamte Kommunikation zwischen Actors baut auf Nachrichten zwischen den beteiligten Actors auf. Wie bereits in Abschnitt \ref{sec:actor:patterns:guaranteedDelivery} diskutiert. kann die Zustellung einer Nachricht nicht, ohne verhältnismäßig viel Aufwand, garantiert werden. Wird in der \textit{TyrolSky} Anwendung eine Nachricht zwischen zwei Actors ausgetauscht wird die Nachricht höchstens einmal zugestellt. Das bedeutet das eine Nachricht unter bestimmten Umständen, wie beispielsweise Netzwerkfehler, nicht zugestellt wird und dies der Versender ohne spezielle Erkennungsmuster nicht mitbekommt. \\
Das Anwendungsumfeld von \textit{TyrolSky} benötigt jedoch auch Kommunikation, wo sichergestellt wird, dass die Zustellung einer Nachricht tatsächlich bis zum Empfänger funktioniert. Dies ist unter anderem der Fall, wenn ein Ticket sich selber storniert. Dabei muss sichergestellt sein, dass die zugehörige Passagierliste über die stornierung Informiert wird, um anschließend den dazugehörigen Sitzplatz für andere wieder freizugegeben. Dafür wurde die Garantierte Nachrichtenzustellung implementiert. Diese Zustellvariante ist eine Implementierung von garantierte \textit{Zustellung mindestens einmal}. \\
Die Ausführung der Garantierten Nachrichtenzustellung basiert auf Implementierung von \textit{Event-Sourcing}, beschrieben unter \ref{sec:implementation:eventSouring}. Dabei wird anstelle der Nachrichtenübermittelung an einen Actor mit den Methoden \textit{Tell} oder \textit{Ask}, siehe Abschnitt \ref{subsec:implementation:akka:cluster}, ein Event beim Sender Actor erzeugt, welches die Zustellung einer Nachricht an einen Actor signalisiert. Nach der erfolgreichen Speicherung des Events mit dem Namen \textit{GuaranteedMessageSent}, wird die Nachricht in einer Wrapper Nachricht vom Typ \textit{GuranteedMessage} zum Empfänger zugstellt. Zusätzlich wird im \textit{State} des Senders abgespeichert, dass diese Nachricht versendet wurde und eine Bestätigung erwartet wird. Trifft innerhalb einer konfigurierbaren Zeit, im Falle von \textit{TyrolSky} sind dies 5 Sekunden, keine Bestätigung für diese Nachricht ein, so wird diese erneut an den Empfänger zugestellt. Durch die Implementierung über \textit{Event Sourcing} werden nicht bestätigte Nachrichten auch nach einem neustart des Actors wieder versandt. \\
Trifft eine Nachricht vom Typ \textit{GuaranteedMessage} ein und wird bearbeitet, so wird eine Bestätigungsnachricht vom Typ \textit{Confirm}, welcher auch Informationen über die Orginal Nachricht enthält, an den Sender der ursprünglichen Nachricht gesendet. Empfängt dieser nun die Nachricht vom Typ \textit{Confirm} so markiert er die Original Nachricht als bestätigt. Somit kann der Sender Bestätigen das die Nachricht an seinem Zielort angekommen ist. Ist eine Zustellung nach mehreren Versuchen trotzdem nicht möglich, kann der Sender der Nachricht entsprechend darauf reagieren, hierfür ist jedoch eine angepasst Implementierung je nach Anwendungsfall notwendig. \\