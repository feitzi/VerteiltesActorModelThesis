\section{Garantierte Nachrichtenübermittelung}
Die gesamte Kommunikation zwischen Actors baut auf Nachrichten auf. Wie bereits in Abschnitt \ref{sec:actor:patterns:guaranteedDelivery} diskutiert, kann die Zustellung einer Nachricht nicht garantiert werden, ohne verhältnismäßig viel Aufwand zu betreiben. Wird in der \textit{TyrolSky} Anwendung eine Nachricht zwischen zwei Actors ausgetauscht, kann die Nachricht höchstens einmal zugestellt werden. Das bedeutet, dass eine Nachricht unter bestimmten Umständen, wie beispielsweise Netzwerkfehler, nicht zugestellt wird und dies der Versender, ohne spezielle Erkennungsmuster, nicht mitbekommt. \\
Das Anwendungsumfeld von \textit{TyrolSky} benötigt  Kommunikation, bei der sichergestellt wird, dass die Zustellung einer Nachricht tatsächlich bis zum Empfänger funktioniert. Dies ist der Fall, wenn ein Ticket sich selber storniert. Dabei muss garantiert werden, dass die zugehörige Passagierliste über die Stornierung informiert wird, um anschließend den dazugehörigen Sitzplatz für andere wieder frei zugegeben. Dazu wurde die Garantierte Nachrichtenzustellung implementiert. Diese Zustellvariante ist eine Implementierung, der in \ref{sec:actor:patterns:guaranteedDelivery} vorgestellten Zustellvariante \textit{garantierte Zustellung mindestens einmal}. \\
Die Ausführung der Garantierten Nachrichtenzustellung basiert auf der Umsetzung des \textit{Event-Sourcings}, beschrieben unter \ref{sec:implementation:eventSouring}. Dabei wird, anstelle der Nachrichtenübermittelung an einen Actor mit den Methoden \textit{Tell} oder \textit{Ask}, siehe Abschnitt \ref{subsec:implementation:akka:cluster}, ein Event beim Sender Actor erzeugt, welcher die Zustellung einer Nachricht an einen Actor signalisiert. Nach der erfolgreichen Speicherung des Events mit dem Namen \textit{GuaranteedMessageSent} wird die Nachricht in einer Wrapper-Nachricht vom Typ \textit{GuranteedMessage} zum Empfänger zugestellt. Zusätzlich wird im \textit{State} des Sender Actors abgespeichert, dass diese Nachricht versendet wurde und vom Sender eine Bestätigung erwartet wird. Trifft innerhalb einer konfigurierbaren Zeit, im Falle von \textit{TyrolSky} sind dies 5 Sekunden, keine Bestätigung für diese Nachricht ein, wird diese erneut an den Empfänger zugestellt. Durch die Implementierung der Zustellvariante mittels \textit{Event Sourcing} werden nicht bestätigte Nachrichten auch nach einem Neustart des Actors wieder versandt. \\
% * <feitzinger.magdalena@gmail.com> 2018-05-19T21:16:45.945Z:
% 
% > Dabei wird, anstelle der Nachrichtenübermittelung an einen Actor mit den Methoden \textit{Tell} oder \textit{Ask}, siehe Abschnitt \ref{subsec:implementation:akka:cluster}, ein Event beim Sender Actor erzeugt, welches die Zustellung einer Nachricht an einen Actor signalisiert. 
% zu verschachtelt! 
% 
% ^.
Trifft eine Nachricht vom Typ \textit{GuaranteedMessage} ein und wird bearbeitet, wird eine Bestätigungsnachricht vom Typ \textit{Confirm}, welcher auch Informationen über die Original-Nachricht enthält, an den Sender der ursprünglichen Nachricht gesendet. Empfängt dieser nun die Nachricht vom Typ \textit{Confirm}, markiert er die Original-Nachricht als bestätigt. Somit kann der Sender bestätigen, dass die Nachricht an seinem Zielort angekommen ist. Ist eine Zustellung nach mehreren Versuchen trotzdem nicht möglich, kann der Sender der Nachricht entsprechend darauf reagieren, hierfür ist eine angepasste Implementierung je nach Anwendungsfall notwendig. \\
% * <feitzinger.magdalena@gmail.com> 2018-05-19T21:21:19.775Z:
% 
% > Trifft eine Nachricht vom Typ \textit{GuaranteedMessage} ein und wird bearbeitet, wird eine Bestätigungsnachricht vom Typ \textit{Confirm}, welcher auch Informationen über die Original-Nachricht enthält, an den Sender der ursprünglichen Nachricht gesendet. Empfängt dieser nun die Nachricht vom Typ \textit{Confirm}, markiert er die Original-Nachricht als bestätigt. Somit kann der Sender bestätigen, dass die Nachricht an seinem Zielort angekommen ist. Ist eine Zustellung nach mehreren Versuchen trotzdem nicht möglich, kann der Sender der Nachricht entsprechend darauf reagieren, hierfür ist eine angepasste Implementierung je nach Anwendungsfall notwendig. \\
% Wortwiederholung "wird" und "bestätigen"
% 
% ^.